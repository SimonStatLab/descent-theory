%% LyX 2.1.3 created this file.  For more info, see http://www.lyx.org/.
%% Do not edit unless you really know what you are doing.
\documentclass[english]{article}
\usepackage[T1]{fontenc}
\usepackage[latin9]{inputenc}
\usepackage{geometry}
\geometry{verbose,tmargin=3cm,bmargin=3cm,lmargin=3cm,rmargin=3cm}
\usepackage{amsmath}
\usepackage{esint}
\usepackage{babel}
\begin{document}

\section{Cross-validation Proof}


\subsubsection*{The problem}

Let $\theta^{*}$ be the true model parameters and suppose we observe
\[
y=g_{\theta^{*}}+\epsilon
\]


where $\epsilon$ are the sub-gaussian random errors. Let $T$ be
the training set and $V$ be the validation set.

Let the loss function for our regression problem be least squares.
We denote it as $\|h\|_{T}^{2}=\frac{1}{n}\sum_{i\in T}h(x_{i})^{2}$
and similarly for $\|\cdot\|_{V}$. 

Consider the joint optimization problem on a training/validation split
to find the best regularization parameter $\lambda$ in $\Lambda$:

\[
\hat{\lambda}=\arg\min_{\lambda\in\Lambda}\frac{1}{2}\|y-g_{\hat{\theta}(\lambda)}\|_{V}^{2}
\]
\[
\hat{\theta}(\lambda)=\arg\min_{\theta}\frac{1}{2}\|y-g_{\theta}\|_{T}^{2}+\lambda\left(P(\theta)+\frac{w}{2}\|\theta\|_{2}^{2}\right)
\]


Here $w>0$ is some fixed parameter. (One should choose this to be
on the order of 1e-15 or smaller. In practice, we can probably just
drop it entirely.)

Let the range of $\Lambda$ be from $[\lambda_{min},\lambda_{max}]$.
Both can grow and shrink at any polynomial rate, e.g. $\mbox{\ensuremath{\lambda}}_{max}=O_{P}(n^{\tau_{max}})$
and $\mbox{\ensuremath{\lambda}}_{min}=O_{P}(n^{-\tau_{min}})$. 

Suppose there is an optimal $\tilde{\lambda}$ s.t. $\|g_{\hat{\theta}(\lambda)}-g^{*}\|$
is some optimal rate $O_{p}(n^{-r})$ where $r\ge1/2$.

We show that $\hat{\lambda}$will converge to $\tilde{\lambda}$ at
an almost-parametric rate or just the optimal rate $O_{p}(n^{-r})$.

The almost-parametric rate is

\[
\|g_{\hat{\theta}(\hat{\lambda})}-g_{\hat{\theta}(\tilde{\lambda})}\|_{V}\le\left(\frac{1-\log w+\kappa\log n}{n}\right)^{1/2}
\]


where $\kappa$ is a constant that depends on known constants, including
$\tau_{max}$ and $\tau_{\min}$.


\subsubsection*{Other Assumptions}
\begin{itemize}
\item Suppose that penalty $P$ is smooth and convex -- its $m$-th derivative
$\nabla_{\theta}^{(m)}P(\theta)$ is defined. Suppose $\|\nabla_{\theta}P(\theta)|_{\hat{\theta}(\lambda)}\|\le O_{p}(n^{k_{1}})$.
For $m\ge2$, suppose the derivative is bounded $\nabla_{\theta}^{(m)}P(\theta)\le K_{m}$.
\item Suppose that the $m$-th derivative $\nabla_{\theta}^{(m)}g_{\theta}$
exists. Suppose $\|\nabla_{\theta}g_{\theta=\hat{\theta}(\lambda)}\|\le O_{p}(n^{g_{1}})$.
Suppose the same things as above for the $m$th derivatives of $g_{\theta}$
wrt $\theta$.
\item Suppose $g_{\theta}$ is Lipschitz with constant $L$ s.t. 
\[
\|g_{\theta_{1}}-g_{\theta_{2}}\|_{n}\le L\left\Vert \theta_{1}-\theta_{2}\right\Vert _{2}
\]

\end{itemize}

\subsubsection*{Proof}

\textbf{Step 1: Find the entropy of the model class $\mathcal{G}_{\lambda}$}

First we show that the entropy $H(u,\mathcal{G}_{\lambda},\|\cdot\|_{V})$
of the class 
\[
\mathcal{G}_{\lambda}=\left\{ \theta_{\lambda}:\lambda\in\Lambda\right\} 
\]
 

is bounded at a near-parametric rate:
\[
H\left(u,\mathcal{G}_{\lambda},\|\cdot\|_{V}\right)\le\log\left(\frac{L}{uw}\right)+\kappa\log n
\]


Using a Taylor expansion, we know that for some $\delta>0$, then
\[
\|\hat{\theta}(\ell)-\hat{\theta}(\ell+\delta)\|=\delta\left\Vert \nabla_{\lambda}\hat{\theta}(\lambda)|_{\lambda=\ell}\right\Vert +O_{P}(\delta^{2})
\]


Since the problem is smooth, we can find the second derivative via
implicit differentiation:

\[
\nabla_{\lambda}\hat{\theta}(\lambda)=\left[\nabla_{\theta}^{2}L_{T}(y,g_{\theta})\right]^{-1}\left(\nabla_{\theta}P(\theta)+w\theta\right)
\]


Note that the Hessian has the form
\[
\nabla_{\theta}^{2}L_{T}(y,g_{\theta})=\nabla_{\theta}^{2}\left(\|y-g_{\theta}\|^{2}+\lambda P(\theta)\right)+\lambda wI
\]
which means its minimum eigenvalue is at least $\lambda w=O_{p}(n^{-\tau_{min}})w$.

From the asumptions, we have that

\begin{eqnarray*}
\left\Vert \nabla_{\lambda}\hat{\theta}(\lambda)\right\Vert  & \le & \left\Vert \left[\nabla_{\theta}^{2}L_{T}(y,g_{\theta})\right]^{-1}\left(\nabla_{\theta}P(\theta)+w\hat{\theta}(\lambda)\right)\right\Vert \\
 & \le & O_{p}(n^{\tau_{min}})w^{-1}\left(O_{p}(n^{k_{1}})+\|\hat{\theta}(\lambda)\|\right)
\end{eqnarray*}


Note that $\|\hat{\theta}(\lambda)\|$ is easily bounded by the solution
to the ridge regression problem:

\[
\hat{\theta}_{r}(\lambda)=\arg\min\|y-g_{\theta}\|^{2}+\lambda\|\theta\|_{2}^{2}
\]


The solution $\hat{\theta}_{r}(\lambda)$ has norm at most $O_{P}(n^{\tau_{\min}})$.
It is straightforward to show that

\[
\|\hat{\theta}(\lambda)\|\le\|\hat{\theta}_{r}(\lambda)\|
\]


Therefore

\[
\left\Vert \nabla_{\lambda}\hat{\theta}(\lambda)\right\Vert =w^{-1}O_{p}(n^{\max(2\tau_{min},\tau_{min}+k_{1})})
\]


Using the Taylor's expansion from above, then we can know bound the
distance between $g_{\hat{\theta}(\ell)}$ and $g_{\hat{\theta}(\ell+\delta)}$

\begin{eqnarray*}
\|g_{\hat{\theta}(\ell)}-g_{\hat{\theta}(\ell+\delta)}\|_{V} & \le & L\left\Vert \hat{\theta}(\ell)-\hat{\theta}(\ell+\delta)\right\Vert _{2}\\
 & = & L\left(\delta w^{-1}O_{p}(n^{\max(2\tau_{min},\tau_{min}+k_{1})})+O_{p}(\delta^{2})\right)
\end{eqnarray*}


Then the covering number is on the order of $n^{\kappa}$ where $\kappa$
grows linearly in $\tau_{min},\tau_{max}$ and $k_{1}$. 
\[
N\left(u,\mathcal{G}_{\lambda},\|\cdot\|_{V}\right)\le\frac{L}{uw}O_{p}(n^{\kappa})\implies H\left(u,\mathcal{G}_{\lambda},\|\cdot\|_{V}\right)\le\log\left(\frac{L}{uw}\right)+\kappa\log n
\]


\textbf{Step 2: Find the basic inequality}

By definition, 
\[
\|y-g_{\hat{\theta}(\hat{\lambda})}\|_{V}^{2}\le\|y-g_{\hat{\theta}(\tilde{\lambda})}\|_{V}^{2}
\]


Rearranging, we get the basic inequality

\[
\|g_{\hat{\theta}(\hat{\lambda})}-g_{\hat{\theta}(\tilde{\lambda})}\|_{V}^{2}\le2\left|\left(\epsilon,g_{\hat{\theta}(\hat{\lambda})}-g_{\hat{\theta}(\tilde{\lambda})}\right)\right|_{V}+\left|\left(g^{*}-g_{\hat{\theta}(\tilde{\lambda})},g_{\hat{\theta}(\hat{\lambda})}-g_{\hat{\theta}(\tilde{\lambda})}\right)\right|_{V}
\]


Clearly if the second term dominates, Cauchy Schwarz (and a symmetrization
argument) give us that $\|g_{\hat{\theta}(\hat{\lambda})}-g_{\hat{\theta}(\tilde{\lambda})}\|$
converges that the same rate as $\|g^{*}-g_{\hat{\theta}(\tilde{\lambda})}\|$.

In the next step we bound the empirical process term.

\textbf{Step 3: Bound the empirical process with high probability}

We apply Corollary 8.3 in Vandegeer to determine the value $\delta$
s.t. $\delta$ bounds the empirical process term with high probability.

Let 
\[
\tilde{\mathcal{G}_{\lambda}}=\left\{ \frac{g_{\hat{\theta}(\lambda)}-g_{\hat{\theta}(\tilde{\lambda})}}{\|g_{\hat{\theta}(\lambda)}-g_{\hat{\theta}(\tilde{\lambda})}\|_{V}}:\lambda\in\Lambda\right\} 
\]


Consequently, 

\begin{eqnarray*}
\int_{0}^{1}H^{1/2}\left(u,\tilde{\mathcal{G}_{\lambda}},\|\cdot\|_{V}\right)du & = & \int_{0}^{1}\left(\log\left(\frac{1}{uw}\right)+\kappa\log n\right)^{1/2}du\\
 & \le & \left(\int_{0}^{1}\log\left(\frac{1}{uw}\right)+\kappa\log ndu\right)^{1/2}\\
 & \le & \left(1-\log w+\kappa\log n\right)^{1/2}
\end{eqnarray*}


By Corollary 8.3, if we choose 
\[
\delta\ge\left(\frac{1-\log w+\kappa\log n}{n}\right)^{1/2}
\]


then

\[
Pr\left(\frac{\left|\left(\epsilon,g_{\hat{\theta}(\hat{\lambda})}-g_{\hat{\theta}(\tilde{\lambda})}\right)\right|_{V}}{\|g_{\hat{\theta}(\lambda)}-g_{\hat{\theta}(\tilde{\lambda})}\|_{V}}\ge\delta\right)\le\exp\left(-n\frac{\delta^{2}}{C^{2}}\right)
\]


\textbf{Step 4: Win}

Returning to the basic inequality in Step 2, we find that in the case
that the empirical process term is bigger, we have

\begin{eqnarray*}
\|g_{\hat{\theta}(\hat{\lambda})}-g_{\hat{\theta}(\tilde{\lambda})}\|_{V} & \le & 4\left|\frac{(\epsilon,g_{\hat{\theta}(\hat{\lambda})}-g_{\hat{\theta}(\tilde{\lambda})})_{V}}{\|g_{\hat{\theta}(\lambda)}-g_{\hat{\theta}(\tilde{\lambda})}\|_{V}}\right|\\
 & \le & 4\left(\frac{1-\log w+\kappa\log n}{n}\right)^{1/2}
\end{eqnarray*}



\section{Lemmas}


\subsection*{Convergence Rate Equivalence between the original regression problem
and the perturbed ridge problem}

Suppose the original regression problem is 
\[
\hat{\beta}(\lambda)=\arg\min_{\beta\in\Theta}\frac{1}{2}\|y-g_{\beta}\|_{T}^{2}+\lambda P(\beta)
\]


and the new perturbed ridge problem is 
\[
\hat{\theta}(\lambda)=\arg\min_{\theta\in\Theta}\frac{1}{2}\|y-g_{\theta}\|_{T}^{2}+\lambda\left(P(\theta)+\frac{w}{2}\|\theta\|_{2}^{2}\right)
\]


Let $\theta^{*}$ be the true model parameters. Suppose there are
constants $K_{0},K_{1}>0$ s.t. 
\[
\frac{w}{2}\|\theta^{*}\|_{2}^{2}\le K_{0}P(\theta^{*})+K_{1}
\]


Then the rate of convergence of $\left\Vert g_{\hat{\beta}(\lambda)}-g^{*}\right\Vert $
is the same as $\left\Vert g_{\hat{\theta}(\lambda)}-g^{*}\right\Vert $
modulo some constant.


\subsubsection*{Proof}

For ease of notation, we will write $\hat{\theta}=\hat{\theta}(\lambda)$
and $\hat{\beta}=\hat{\beta}(\lambda)$.

By definition, 
\begin{eqnarray*}
\frac{1}{2}\|y-g_{\hat{\theta}}\|^{2}+\tilde{\lambda}\left(P(\hat{\theta})+\frac{w}{2}\|\hat{\theta}\|^{2}\right) & \le & \frac{1}{2}\|y-g_{\theta^{*}}\|^{2}+\tilde{\lambda}\left(P(\theta^{*})+\frac{w}{2}\|\theta^{*}\|^{2}\right)\\
 & \le & \frac{1}{2}\|y-g_{\theta^{*}}\|^{2}+\tilde{\lambda}(1+K_{0})P(\theta^{*})+\tilde{\lambda}K_{1}
\end{eqnarray*}


Therefore
\[
\frac{1}{2}\|y-g_{\hat{\theta}}\|^{2}+\tilde{\lambda}P(\hat{\theta})\le\frac{1}{2}\|y-g_{\theta^{*}}\|^{2}+\tilde{\lambda}(1+K_{0})P(\theta^{*})+\tilde{\lambda}K_{1}
\]


Notice that this inequality is very similar to the inequality from
the original regression problem

\[
\frac{1}{2}\|y-g_{\hat{\beta}}\|^{2}+\tilde{\lambda}P(\hat{\beta})\le\frac{1}{2}\|y-g_{\theta^{*}}\|^{2}+\tilde{\lambda}P(\theta^{*})
\]


Therefore the same arguments to prove the convergence rate of $\left\Vert g_{\hat{\beta}(\lambda)}-g^{*}\right\Vert $
give the same convergence rate for $\left\Vert g_{\hat{\theta}(\lambda)}-g^{*}\right\Vert $.
(Example: refer to Thrm 10.2 in Vandegeer)


\section{Examples}


\subsubsection*{Ridge Regression}

\[
\hat{\theta}(\lambda)=\arg\min_{\theta}\frac{1}{2}\|y-X\theta\|_{T}^{2}+\frac{\lambda}{2}\|\theta\|^{2}
\]


The Hessian is 
\[
X^{T}X+\lambda I
\]
so its minimum eigenvalue is $\lambda$.

The derivative of the penalty wrt $\theta$ is 
\[
\nabla_{\theta}P(\theta)=\theta
\]


Since $\theta=(X^{T}X+\lambda I)^{-1}X^{T}y$, the derivative has
bounded norm $\|\nabla_{\theta}P(\theta)\|=O_{P}(n^{\tau_{min}})$.

All other assumptions are obviously satisfied. Note that in this problem,
we can drop the tiny additional ridge penalty entirely.


\subsubsection*{Smoothing Splines with a Sobolev Penalty}

The optimization problem can be formulated as

\[
\hat{\theta}(\lambda)=\arg\min_{\theta}\frac{1}{2}\|y-\theta\|_{T}^{2}+\frac{\lambda}{2}\theta^{T}K\theta
\]


where $K=N^{-T}\Omega N$, $N$ is the normalized B-splines evaluated
at the input points, and $\Omega$ has entries $\Omega_{ij}=\int b_{i}"(x)b_{j}"(x)dx$.

Assume the input points $i/n$ for $i=1:n$ and we fit $y$ with cubic
B-splines. Then $K=DC^{-1}Dn^{3}$ where $D$ is the (universal, non-data-dependent)
second-order discrete diference operator and $C$ is a tridiagonal
matrix with diagonal elements equal to 2/3 and off-diagonal elements
equal to 1/6.

The Hessian is 
\[
I+\lambda K
\]
so its minimum eigenvalue is 1.

The derivative of the penalty wrt $\theta$ is 
\[
\nabla_{\theta}P(\theta)=K\theta
\]


The maximum eigenvalue of $K$ is on the order of $O_{p}(n^{-3})$.
Clearly$\|\theta\|$ can be bounded by $\|y\|$ modulo a constant.
So $\|\nabla_{\theta}P(\theta)\|=O_{P}(n^{3})$. Also, $\nabla_{\theta}^{2}P(\theta)=K$
so its norm is shrinking at a rate of $O_{P}(n^{-3})$.

All other assumptions are obviously satisfied. Note that in this problem,
we can drop the tiny additional ridge penalty entirely.

\textbf{Lasso (with a tiny ridge)}

\[
\hat{\theta}(\lambda)=\arg\min_{\theta}\frac{1}{2}\|y-X\theta\|_{T}^{2}+\lambda\left(\|\theta\|_{1}+\frac{w}{2}\|\theta\|_{2}^{2}\right)
\]


With modifications to the proof above, we can probably show that the
proof carries through for penalties that are smooth almost everywhere.
The main tricky part is dealing with the fact that the Hessian with
respect to the differentiable space changes in size for different
values of $\lambda$.

Anyhow, by a hand-wavy argument, we have that the Hessian matrix with
respect to the differentiable space $S_{\lambda}$ is 
\[
X_{S_{\lambda}}^{T}X_{S_{\lambda}}+\lambda wI_{S_{\lambda}}
\]
 so its minimum eigenvalue is $w\lambda$.

The lasso penalty clearly satisfies our assumptions since its derivative
$\nabla_{\theta}\|\theta\|_{1}=sgn(\theta)$ has bounded norm $\|\nabla_{\theta}\|\theta\|_{1}\|\le p$.

All other assumptions are satisfied.
\end{document}
