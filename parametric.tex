%% LyX 2.1.3 created this file.  For more info, see http://www.lyx.org/.
%% Do not edit unless you really know what you are doing.
\documentclass[english]{article}
\usepackage[T1]{fontenc}
\usepackage[latin9]{inputenc}
\usepackage[landscape]{geometry}
\geometry{verbose,tmargin=3cm,bmargin=3cm,lmargin=3cm,rmargin=3cm}
\usepackage{amsmath}
\usepackage{amssymb}
\usepackage{esint}
\usepackage{babel}
\begin{document}

\title{Proofs for Smoothness of Parametric Regression Models}

\maketitle

\section*{Intro}

In this document, we consider parametric regression models $g(\cdot|\boldsymbol{\theta})$
where $\boldsymbol{\theta}\in\mathbb{R}^{p}$. Throughout, we will
suppose that the projection of the true model into the parametric
model space is $g(x|\boldsymbol{\theta}^{*})$.

We are interested in establishing inequalities of the form 
\[
\|\boldsymbol{\hat{\theta}_{\lambda^{(1)}}}-\boldsymbol{\hat{\theta}_{\lambda^{(2)}}}\|_{2}\le C\|\boldsymbol{\lambda^{(2)}}-\boldsymbol{\lambda^{(1)}}\|_{2}
\]


If the functions are Lipschitz in their parameterization, we will
also be able to bound the distance between the actual functions. That
is, if there are constants $L>0$ and $r\in\mathbb{R}$, such that
for all $\boldsymbol{\theta_{1},\theta_{2}}$ 
\[
\|g(\cdot|\boldsymbol{\theta}_{1})-g(\cdot|\boldsymbol{\theta}_{2})\|_{\infty}\le Lp^{r}\|\boldsymbol{\theta}_{1}-\boldsymbol{\theta}_{2}\|_{2}
\]


Then
\[
\|g(\cdot|\boldsymbol{\hat{\theta}_{\lambda^{(1)}}})-g(\cdot|\boldsymbol{\hat{\theta}_{\lambda^{(2)}}})\|_{\infty}\le Lp^{r}C\|\boldsymbol{\lambda^{(2)}}-\boldsymbol{\lambda^{(1)}}\|_{2}
\]



\subsubsection*{Document Outline}

First, we consider smooth training criterions and prove smoothness
for two parametric regression examples:
\begin{enumerate}
\item Multiple penalties for a single model
\[
\hat{\boldsymbol{\theta}}_{\lambda}=\arg\min_{\theta\in\mathbb{R}^{p}}\frac{1}{2}\|y-g(\cdot|\boldsymbol{\theta})\|_{T}^{2}+\sum_{j=1}^{J}\lambda_{j}\left(P_{j}(\boldsymbol{\theta})+\frac{w}{2}\|\boldsymbol{\theta}\|_{2}^{2}\right)
\]

\item Additive model
\[
\hat{\boldsymbol{\theta}}_{\lambda}=\arg\min_{\theta\in\mathbb{R}^{p}}\frac{1}{2}\|y-\sum_{j=1}^{J}g_{j}(\cdot|\boldsymbol{\theta}_{j})\|_{T}^{2}+\sum_{j=1}^{J}\lambda_{j}\left(P_{j}(\boldsymbol{\theta}_{j})+\frac{w}{2}\|\boldsymbol{\theta}_{j}\|_{2}^{2}\right)
\]

\end{enumerate}
Then we will extend these results to non-smooth penalty functions.

Finally we will consider examples of parametric penalty functions.
It includes a deep dive into the Sobolev penalty.


\section{Multiple smooth penalties for a single model}

The function class of interest are the minimizers of the penalized
least squares criterion: 
\[
\mathcal{G}(T)=\left\{ \hat{\boldsymbol{\theta}}_{\lambda}=\arg\min_{\theta\in\mathbb{R}^{p}}\frac{1}{2}\|y-g(\cdot|\boldsymbol{\theta})\|_{T}^{2}+\sum_{j=1}^{J}\lambda_{j}\left(P_{j}(\boldsymbol{\theta})+\frac{w}{2}\|\boldsymbol{\theta}\|_{2}^{2}\right):\boldsymbol{\lambda}\in\Lambda\right\} 
\]


where $\Lambda=[\lambda_{min},\lambda_{max}]^{J}$ and $w>0$ is a
fixed constant.

Suppose that the penalties and the function $g(x|\boldsymbol{\theta})$
are smooth and convex wrt $\boldsymbol{\theta}$: 
\begin{itemize}
\item Suppose that $\nabla_{\theta}^{2}P_{j}(\boldsymbol{\theta})$ are
PSD matrices for all $j=1,...,J$. 
\item Suppose that $\nabla_{\theta}^{2}g(x|\boldsymbol{\theta})$ are PSD
matrices for all $x$.
\end{itemize}
\textbf{Primary Assumption} (rephrase?) : Suppose there is some $K>0$
such that for all $j=1,...,J$ and any $\boldsymbol{\theta,\beta}$,
we have

\[
\left|\frac{\partial}{\partial m}P_{j}\left(\boldsymbol{\theta}+m\boldsymbol{\beta}\right)\right|\le K\|\boldsymbol{\beta}\|_{2}
\]


(This is essentially bounding the spectrum of the penalty function)

\textbf{Result}

Let 
\[
C=\frac{1}{2}\|y-g(\cdot|\boldsymbol{\theta}^{*})\|_{T}^{2}+\lambda_{max}\sum_{j=1}^{J}\left(P_{j}(\boldsymbol{\theta}^{*})+\frac{w}{2}\|\boldsymbol{\theta}^{*}\|_{2}^{2}\right)
\]


Then for any $\boldsymbol{\lambda^{(1)},\lambda^{(2)}}\in\Lambda$
we have
\[
\|\boldsymbol{\hat{\theta}_{\lambda^{(1)}}}-\boldsymbol{\hat{\theta}_{\lambda^{(2)}}}\|_{2}\le\|\boldsymbol{\lambda^{(2)}}-\boldsymbol{\lambda^{(1)}}\|_{2}\left(w\sqrt{J}\lambda_{min}\right)^{-1}\left(K+w\sqrt{\frac{2}{J\lambda_{min}w}\left(1+\frac{J\lambda_{max}}{\lambda_{min}}\right)C}\right)
\]



\subsubsection*{Proof}

Consider any $\boldsymbol{\lambda^{(1)},\lambda^{(2)}}\in\Lambda$.
Let $\boldsymbol{\beta}=\hat{\boldsymbol{\theta}}_{\lambda^{(2)}}-\hat{\boldsymbol{\theta}}_{\lambda^{(1)}}$.

Define 
\[
\hat{m}_{\boldsymbol{\beta}}(\boldsymbol{\lambda})=\arg\min_{m\in\mathbb{R}}\frac{1}{2}\left\Vert y-g(\cdot|\boldsymbol{\hat{\theta}_{\lambda^{(1)}}}+m\boldsymbol{\beta})\right\Vert _{T}^{2}+\sum_{j=1}^{J}\lambda_{j}\left(P_{j}(\boldsymbol{\hat{\theta}_{\lambda^{(1)}}}+m\boldsymbol{\beta})+\frac{w}{2}\|\boldsymbol{\hat{\theta}_{\lambda^{(1)}}}+m\boldsymbol{\beta}\|_{2}^{2}\right)
\]


By definition, we know that $\hat{m}_{\beta}(\boldsymbol{\lambda^{(2)}})=1$
and $\hat{m}_{\beta}(\boldsymbol{\lambda^{(1)}})=0$.

\textbf{1. We calculate $\nabla_{\lambda}\hat{m}_{\boldsymbol{\beta}}(\boldsymbol{\lambda})$
using the implicit differentiation trick.}

By the KKT conditions, we have 
\[
\left.\frac{\partial}{\partial m}\left(\frac{1}{2}\left\Vert y-g(\cdot|\boldsymbol{\hat{\theta}_{\lambda^{(1)}}}+m\boldsymbol{\beta})\right\Vert _{T}^{2}+\sum_{j=1}^{J}\lambda_{j}P_{j}(\boldsymbol{\hat{\theta}_{\lambda^{(1)}}}+m\boldsymbol{\beta})\right)+\sum_{j=1}^{J}\lambda_{j}w\langle\boldsymbol{\beta},\boldsymbol{\hat{\theta}_{\lambda^{(1)}}}+m\boldsymbol{\beta}\rangle\right|_{m=\hat{m}_{\beta}(\boldsymbol{\lambda})}=0
\]


Now we implicitly differentiate with respect to $\lambda_{\ell}$
for $\ell=1,2,...,J$

\[
\frac{\partial}{\partial\lambda_{\ell}}\left\{ \left.\left[\frac{\partial}{\partial m}\left(\frac{1}{2}\left\Vert y-g(\cdot|\boldsymbol{\hat{\theta}_{\lambda^{(1)}}}+m\boldsymbol{\beta})\right\Vert _{T}^{2}+\sum_{j=1}^{J}\lambda_{j}P_{j}(\boldsymbol{\hat{\theta}_{\lambda^{(1)}}}+m\boldsymbol{\beta})\right)+\sum_{j=1}^{J}\lambda_{j}w\langle\boldsymbol{\beta},\boldsymbol{\hat{\theta}_{\lambda^{(1)}}}+m\boldsymbol{\beta}\rangle\right]\right|_{m=\hat{m}_{\beta}(\boldsymbol{\lambda})}\right\} =0
\]


By the product rule and chain rule, we have 

\[
\left.\left\{ \left[\frac{\partial^{2}}{\partial m^{2}}\left(\frac{1}{2}\left\Vert y-g(\cdot|\boldsymbol{\hat{\theta}_{\lambda^{(1)}}}+m\boldsymbol{\beta})\right\Vert _{T}^{2}+\sum_{j=1}^{J}\lambda_{j}P_{j}(\boldsymbol{\hat{\theta}_{\lambda^{(1)}}}+m\boldsymbol{\beta})\right)+\sum_{j=1}^{J}\lambda_{j}w\|\boldsymbol{\beta}\|_{2}^{2}\right]\frac{\partial}{\partial\lambda_{\ell}}\hat{m}_{\beta}(\boldsymbol{\lambda})+\frac{\partial}{\partial m}P_{\ell}(\boldsymbol{\hat{\theta}_{\lambda^{(1)}}}+m\boldsymbol{\beta})+w\langle\boldsymbol{\beta},\boldsymbol{\hat{\theta}_{\lambda^{(1)}}}+m\boldsymbol{\beta}\rangle\right\} \right|_{m=\hat{m}_{\beta}(\boldsymbol{\lambda})}=0
\]


Rearranging, for every $\ell=1,...,J$, we get

\[
\frac{\partial}{\partial\lambda_{\ell}}\hat{m}_{\boldsymbol{\beta}}(\boldsymbol{\lambda})=\left.-\left[\frac{\partial^{2}}{\partial m^{2}}\left(\frac{1}{2}\left\Vert y-g(\cdot|\boldsymbol{\hat{\theta}_{\lambda^{(1)}}}+m\boldsymbol{\beta})\right\Vert _{T}^{2}+\sum_{j=1}^{J}\lambda_{j}P_{j}(\boldsymbol{\hat{\theta}_{\lambda^{(1)}}}+m\boldsymbol{\beta})\right)+\sum_{j=1}^{J}\lambda_{j}w\|\boldsymbol{\beta}\|_{2}^{2}\right]^{-1}\left[\frac{\partial}{\partial m}P_{\ell}(\boldsymbol{\hat{\theta}_{\lambda^{(1)}}}+m\boldsymbol{\beta})+w\langle\boldsymbol{\beta},\boldsymbol{\hat{\theta}_{\lambda^{(1)}}}+m\boldsymbol{\beta}\rangle\right]\right|_{m=\hat{m}_{\beta}(\boldsymbol{\lambda})}
\]


In vector notation, we have 
\[
\nabla_{\lambda}\hat{m}_{\boldsymbol{\beta}}(\boldsymbol{\lambda})=\left.-\left[\frac{\partial^{2}}{\partial m^{2}}\left(\frac{1}{2}\left\Vert y-g(\cdot|\boldsymbol{\hat{\theta}_{\lambda^{(1)}}}+m\boldsymbol{\beta})\right\Vert _{T}^{2}+\sum_{j=1}^{J}\lambda_{j}P_{j}(\boldsymbol{\hat{\theta}_{\lambda^{(1)}}}+m\boldsymbol{\beta})\right)+\sum_{j=1}^{J}\lambda_{j}w\|\boldsymbol{\beta}\|_{2}^{2}\right]^{-1}\left[\nabla_{m}P(\boldsymbol{\hat{\theta}_{\lambda^{(1)}}}+m\boldsymbol{\beta})+w\langle\boldsymbol{\beta},\boldsymbol{\hat{\theta}_{\lambda^{(1)}}}+m\boldsymbol{\beta}\rangle\boldsymbol{1}\right]\right|_{m=\hat{m}_{\beta}(\boldsymbol{\lambda})}
\]


where $\nabla_{m}P(\boldsymbol{\hat{\theta}_{\lambda^{(1)}}}+m\boldsymbol{\beta})$
is the $J$-dimensional vector

\[
\nabla_{m}P(\boldsymbol{\hat{\theta}_{\lambda^{(1)}}}+m\boldsymbol{\beta})=\left[\begin{array}{c}
\frac{\partial}{\partial m}P_{1}(\boldsymbol{\hat{\theta}_{\lambda^{(1)}}}+m\boldsymbol{\beta})\\
...\\
\frac{\partial}{\partial m}P_{J}(\boldsymbol{\hat{\theta}_{\lambda^{(1)}}}+m\boldsymbol{\beta})
\end{array}\right]
\]


\textbf{2. Bound $\|\nabla_{\lambda}\hat{m}_{\boldsymbol{\beta}}(\boldsymbol{\lambda})\|$}

\textbf{Bounding the first multiplicand:}

The first multiplicand is bounded by 
\[
\left|\frac{\partial^{2}}{\partial m^{2}}\left(\frac{1}{2}\left\Vert y-g(\cdot|\boldsymbol{\hat{\theta}_{\lambda^{(1)}}}+m\boldsymbol{\beta})\right\Vert _{T}^{2}+\sum_{j=1}^{J}\lambda_{j}P_{j}(\boldsymbol{\hat{\theta}_{\lambda^{(1)}}}+m\boldsymbol{\beta})\right)+\sum_{j=1}^{J}\lambda_{j}w\|\boldsymbol{\beta}\|_{2}^{2}\right|^{-1}\le\left(wJ\lambda_{min}\|\boldsymbol{\beta}\|_{2}^{2}\right)^{-1}
\]


since the mean squared error and the penalty functions are convex.

\textbf{Bounding the second multiplicand:}

The first summand in the second multiplicand is bounded by assumption
\[
\left|\frac{\partial}{\partial m}P_{\ell}(\boldsymbol{\hat{\theta}_{\lambda^{(1)}}}+m\boldsymbol{\beta})\right|\le K\|\boldsymbol{\beta}\|_{2}
\]


The second summand in the second multiplicand is bounded by
\begin{eqnarray}
\left|w\langle\boldsymbol{\beta},\boldsymbol{\hat{\theta}_{\lambda^{(1)}}}+\hat{m}_{\beta}(\boldsymbol{\lambda})\boldsymbol{\beta}\rangle\right| & \le & w\|\boldsymbol{\beta}\|_{2}\|\boldsymbol{\hat{\theta}_{\lambda^{(1)}}}+\hat{m}_{\beta}(\boldsymbol{\lambda})\boldsymbol{\beta}\|_{2}
\end{eqnarray}


We need to bound $\|\boldsymbol{\hat{\theta}_{\lambda^{(1)}}}+\hat{m}_{\beta}(\boldsymbol{\lambda})\boldsymbol{\beta}\|_{2}$.
By definition of $\hat{m}_{\beta}(\boldsymbol{\lambda})$ , 
\begin{eqnarray*}
\sum_{j=1}^{J}\lambda_{j}\frac{w}{2}\|\boldsymbol{\hat{\theta}_{\lambda^{(1)}}}+\hat{m}_{\beta}(\boldsymbol{\lambda})\boldsymbol{\beta}\|_{2}^{2} & \le & \frac{1}{2}\|y-g(\cdot|\boldsymbol{\hat{\theta}_{\lambda^{(1)}}})\|_{T}^{2}+\sum_{j=1}^{J}\lambda_{j}\left(P_{j}(\boldsymbol{\hat{\theta}_{\lambda^{(1)}}})+\frac{w}{2}\|\boldsymbol{\hat{\theta}_{\lambda^{(1)}}}\|_{2}^{2}\right)\\
 & = & \frac{1}{2}\|y-g(\cdot|\boldsymbol{\hat{\theta}_{\lambda^{(1)}}})\|_{T}^{2}+\sum_{j=1}^{J}\lambda_{j}^{(1)}\left(P_{j}(\boldsymbol{\hat{\theta}_{\lambda^{(1)}}})+\frac{w}{2}\|\boldsymbol{\hat{\theta}_{\lambda^{(1)}}}\|_{2}^{2}\right)+\sum_{j=1}^{J}\left(\lambda_{j}-\lambda_{j}^{(1)}\right)\left(P_{j}(\boldsymbol{\hat{\theta}_{\lambda^{(1)}}})+\frac{w}{2}\|\boldsymbol{\hat{\theta}_{\lambda^{(1)}}}\|_{2}^{2}\right)
\end{eqnarray*}


To bound the first part of the right hand side, use the definition
of $\hat{\boldsymbol{\theta}}_{\lambda^{(1)}}$: 
\begin{eqnarray*}
\frac{1}{2}\|y-g(\cdot|\boldsymbol{\hat{\theta}_{\lambda^{(1)}}})\|_{T}^{2}+\sum_{j=1}^{J}\lambda_{j}^{(1)}\left(P_{j}(\boldsymbol{\hat{\theta}_{\lambda^{(1)}}})+\frac{w}{2}\|\boldsymbol{\hat{\theta}_{\lambda^{(1)}}}\|_{2}^{2}\right) & \le & \frac{1}{2}\|y-g(\cdot|\boldsymbol{\theta}^{*})\|_{T}^{2}+\sum_{j=1}^{J}\lambda_{j}^{(1)}\left(P_{j}(\boldsymbol{\theta}^{*})+\frac{w}{2}\|\boldsymbol{\theta}^{*}\|_{2}^{2}\right)\\
 & \le & \frac{1}{2}\|y-g(\cdot|\boldsymbol{\theta}^{*})\|_{T}^{2}+\lambda_{max}\sum_{j=1}^{J}\left(P_{j}(\boldsymbol{\theta}^{*})+\frac{w}{2}\|\boldsymbol{\theta}^{*}\|_{2}^{2}\right)\\
 & = & C
\end{eqnarray*}


To bound the second part of the right hand side, note that 
\begin{eqnarray*}
\sum_{j=1}^{J}\left(\lambda_{j}-\lambda_{j}^{(1)}\right)\left(P_{j}(\boldsymbol{\hat{\theta}_{\lambda^{(1)}}})+\frac{w}{2}\|\boldsymbol{\hat{\theta}_{\lambda^{(1)}}}\|_{2}^{2}\right) & \le & \sum_{j=1}^{J}\left(\lambda_{j}-\lambda_{j}^{(1)}\right)\left[\max_{k=1:J}P_{k}(\boldsymbol{\hat{\theta}_{\lambda^{(1)}}})+\frac{w}{2}\|\hat{\boldsymbol{\theta}}_{\lambda^{(1)}}\|_{2}^{2}\right]\\
 & \le & J\lambda_{max}\left[\max_{k=1:J}P_{k}(\boldsymbol{\hat{\theta}_{\lambda^{(1)}}})+\frac{w}{2}\|\hat{\boldsymbol{\theta}}_{\lambda^{(1)}}\|_{2}^{2}\right]
\end{eqnarray*}


Combining the above three inequalities, we get
\begin{equation}
\sum_{j=1}^{J}\lambda_{j}\frac{w}{2}\|\boldsymbol{\hat{\theta}_{\lambda^{(1)}}}+\hat{m}_{\beta}(\boldsymbol{\lambda})\boldsymbol{\beta}\|_{2}^{2}\le C+J\lambda_{max}\left[\max_{k=1:J}P_{k}(\boldsymbol{\hat{\theta}_{\lambda^{(1)}}})+\frac{w}{2}\|\hat{\boldsymbol{\theta}}_{\lambda^{(1)}}\|_{2}^{2}\right]
\end{equation}


To bound $\max_{k=1:J}P_{k}(\boldsymbol{\hat{\theta}_{\lambda^{(1)}}})+\frac{w}{2}\|\hat{\boldsymbol{\theta}}_{\lambda^{(1)}}\|_{2}^{2}$,
we note that by the definition of $\hat{\boldsymbol{\theta}}_{\lambda^{(1)}}$,
we have 

\begin{eqnarray*}
\sum_{j=1}^{J}\lambda_{j}^{(1)}\left(P_{j}(\boldsymbol{\hat{\theta}_{\lambda^{(1)}}})+\frac{w}{2}\|\hat{\boldsymbol{\theta}}_{\lambda^{(1)}}\|_{2}^{2}\right) & \le & \frac{1}{2}\|y-g(\cdot|\boldsymbol{\theta}^{*})\|_{T}^{2}+\sum_{j=1}^{J}\lambda_{j}^{(1)}\left(P_{j}(\boldsymbol{\theta}^{*})+\frac{w}{2}\|\boldsymbol{\theta}^{*}\|_{2}^{2}\right)\\
 & \le & C
\end{eqnarray*}


Therefore
\begin{equation}
\max_{k=1:J}P_{k}(\boldsymbol{\hat{\theta}_{\lambda^{(1)}}})+\frac{w}{2}\|\hat{\boldsymbol{\theta}}_{\lambda^{(1)}}\|_{2}^{2}\le\frac{C}{\lambda_{min}}
\end{equation}


Plugging (3) into (2) above, we get 
\begin{eqnarray}
\sum_{j=1}^{J}\lambda_{j}\frac{w}{2}\|\boldsymbol{\hat{\theta}_{\lambda^{(1)}}}+\hat{m}_{\beta}(\boldsymbol{\lambda})\boldsymbol{\beta}\|_{2}^{2} & \le & \left(1+\frac{J\lambda_{max}}{\lambda_{min}}\right)C
\end{eqnarray}


We can combine (4) with the fact that 
\[
J\lambda_{min}\frac{w}{2}\|\boldsymbol{\hat{\theta}_{\lambda^{(1)}}}+\hat{m}_{\beta}(\boldsymbol{\lambda})\boldsymbol{\beta}\|_{2}^{2}\le\sum_{j=1}^{J}\lambda_{j}\frac{w}{2}\|\boldsymbol{\hat{\theta}_{\lambda^{(1)}}}+\hat{m}_{\beta}(\boldsymbol{\lambda})\boldsymbol{\beta}\|_{2}^{2}
\]


to get
\[
\|\boldsymbol{\hat{\theta}_{\lambda^{(1)}}}+\hat{m}_{\beta}(\boldsymbol{\lambda})\boldsymbol{\beta}\|_{2}\le\sqrt{\frac{2}{J\lambda_{min}w}\left(1+\frac{J\lambda_{max}}{\lambda_{min}}\right)C}
\]


Plug the inequality above into (1) to get 
\[
w\langle\boldsymbol{\beta},\boldsymbol{\hat{\theta}_{\lambda^{(1)}}}+\hat{m}_{\beta}(\boldsymbol{\lambda})\boldsymbol{\beta}\rangle\le w\|\boldsymbol{\beta}\|_{2}\sqrt{\frac{2}{J\lambda_{min}w}\left(1+\frac{J\lambda_{max}}{\lambda_{min}}\right)C}
\]


Finally we have bounded the derivative of $\frac{\partial}{\partial\lambda_{\ell}}\hat{m}_{\beta}(\boldsymbol{\lambda})$.
For every $\ell=1,...,J$, we have 
\begin{eqnarray*}
\left|\frac{\partial}{\partial\lambda_{\ell}}\hat{m}_{\beta}(\boldsymbol{\lambda})\right| & \le & \left(wJ\lambda_{min}\|\boldsymbol{\beta}\|_{2}^{2}\right)^{-1}\left(K\|\boldsymbol{\beta}\|_{2}+w\|\boldsymbol{\beta}\|_{2}\sqrt{\frac{2}{J\lambda_{min}w}\left(1+\frac{J\lambda_{max}}{\lambda_{min}}\right)C}\right)\\
 & = & \left(wJ\lambda_{min}\|\boldsymbol{\beta}\|_{2}\right)^{-1}\left(K+w\sqrt{\frac{2}{J\lambda_{min}w}\left(1+\frac{J\lambda_{max}}{\lambda_{min}}\right)C}\right)
\end{eqnarray*}


We can sum up these bounds to bound the norm of the gradient $\nabla_{\lambda}\hat{m}_{\beta}(\boldsymbol{\lambda})$:
\begin{eqnarray*}
\left\Vert \nabla_{\lambda}\hat{m}_{\beta}(\boldsymbol{\lambda})\right\Vert  & = & \sqrt{\sum_{\ell=1}^{J}\left(\frac{\partial}{\partial\lambda_{\ell}}\hat{m}_{\beta}(\boldsymbol{\lambda})\right)^{2}}\\
 & \le & \left(w\lambda_{min}\sqrt{J}\|\boldsymbol{\beta}\|_{2}\right)^{-1}\left(K+w\sqrt{\frac{2}{J\lambda_{min}w}\left(1+\frac{J\lambda_{max}}{\lambda_{min}}\right)C}\right)
\end{eqnarray*}


\textbf{3. Apply Mean Value Theorem}

Since the training criterion is smooth, then $\hat{m}_{\boldsymbol{\beta}}(\boldsymbol{\lambda})$
is continuous and differentiable over the line segment $\left\{ \alpha\boldsymbol{\lambda^{(1)}}+(1-\alpha)\boldsymbol{\lambda^{(2)}}:\alpha\in[0,1]\right\} $.
Therefore by MVT, there is some $\alpha\in(0,1)$ such that 
\begin{eqnarray*}
\left|\hat{m}_{\beta}(\boldsymbol{\lambda^{(2)}})-\hat{m}_{\beta}(\boldsymbol{\lambda^{(1)}})\right| & = & \left|\left.\left\langle \boldsymbol{\lambda^{(2)}}-\boldsymbol{\lambda^{(1)}},\nabla_{\lambda}\hat{m}_{\beta}(\boldsymbol{\lambda})\right\rangle \right|_{\boldsymbol{\lambda}=\alpha\boldsymbol{\lambda^{(1)}}+(1-\alpha)\boldsymbol{\lambda^{(2)}}}\right|\\
 & \le & \|\boldsymbol{\lambda^{(2)}}-\boldsymbol{\lambda^{(1)}}\|_{2}\left\Vert \left.\nabla_{\lambda}\hat{m}_{\beta}(\boldsymbol{\lambda})\right|_{\boldsymbol{\lambda}=\alpha\boldsymbol{\lambda^{(1)}}+(1-\alpha)\boldsymbol{\lambda^{(2)}}}\right\Vert \\
 & \le & \|\boldsymbol{\lambda^{(2)}}-\boldsymbol{\lambda^{(1)}}\|_{2}\left(w\sqrt{J}\lambda_{min}\|\boldsymbol{\beta}\|_{2}\right)^{-1}\left(K+w\sqrt{\frac{2}{J\lambda_{min}w}\left(1+\frac{J\lambda_{max}}{\lambda_{min}}\right)C}\right)
\end{eqnarray*}


Recall that $\hat{m}_{\beta}(\boldsymbol{\lambda^{(2)}})-\hat{m}_{\beta}(\boldsymbol{\lambda^{(1)}})=1$.
Rearranging, we get
\[
\|\boldsymbol{\beta}\|_{2}=\|\boldsymbol{\hat{\theta}_{\lambda^{(1)}}}-\boldsymbol{\hat{\theta}_{\lambda^{(2)}}}\|_{2}\le\|\boldsymbol{\lambda^{(2)}}-\boldsymbol{\lambda^{(1)}}\|_{2}\left(w\sqrt{J}\lambda_{min}\right)^{-1}\left(K+w\sqrt{\frac{2}{J\lambda_{min}w}\left(1+\frac{J\lambda_{max}}{\lambda_{min}}\right)C}\right)
\]



\section{Additive Model}

The function class of interest are the minimizers of the penalized
least squares criterion: 
\[
\mathcal{G}(T)=\left\{ \hat{\boldsymbol{\theta}}_{\lambda}=\arg\min_{\theta\in\mathbb{R}^{p}}\frac{1}{2}\left\Vert y-\sum_{j=1}^{J}g_{j}(\cdot|\boldsymbol{\theta}_{j})\right\Vert _{T}^{2}+\sum_{j=1}^{J}\lambda_{j}\left(P_{j}(\boldsymbol{\theta}_{j})+\frac{w}{2}\|\boldsymbol{\theta}_{j}\|_{2}^{2}\right):\boldsymbol{\lambda}\in\Lambda\right\} 
\]


where $\Lambda=[\lambda_{min},\lambda_{max}]^{J}$.

Suppose that the penalties and the function $g_{j}(x|\boldsymbol{\theta}_{j})$
is convex wrt $\boldsymbol{\theta}_{j}$: $\nabla_{\boldsymbol{\theta}_{j}}^{2}P_{j}(\boldsymbol{\theta}_{j})$
for all $j=1,...,J$ and $\nabla_{\boldsymbol{\theta}_{j}}^{2}g_{j}(x|\boldsymbol{\theta}_{j})$
are PSD matrices.

Suppose there is some constant $K>0$ such that for all $j=1,...,J$
and all $\boldsymbol{\beta},\boldsymbol{\theta}$,

\[
\left|\frac{\partial}{\partial m}P_{j}(\boldsymbol{\theta}+m\boldsymbol{\beta})\right|\le K\|\boldsymbol{\beta}\|_{2}
\]


(This is essentially bounding the spectrum of the penalty function)

Let 
\[
C=\frac{1}{2}\left\Vert y-\sum_{j=1}^{J}g_{j}(\cdot|\boldsymbol{\theta}_{j}^{*})\right\Vert +\lambda_{max}\sum_{j=1}^{J}\left(P_{j}(\boldsymbol{\theta}_{j}^{*})+\frac{w}{2}\|\boldsymbol{\theta}_{j}^{*}\|_{2}^{2}\right)
\]


Then for any $\boldsymbol{\lambda^{(1)},\lambda^{(2)}}\in\Lambda$
we have for all $j=1,...,J$
\[
\|\boldsymbol{\theta}_{\lambda^{(1)},j}-\boldsymbol{\theta}_{\lambda^{(2)},j}\|\le\left\Vert \boldsymbol{\lambda}^{(2)}-\boldsymbol{\lambda}^{(1)}\right\Vert \left(K+w\sqrt{\frac{2C}{\lambda_{min}w}\left(1+\frac{J\lambda_{max}}{\lambda_{min}}\right)}\right)\lambda_{min}^{-1}w^{-1}
\]



\subsubsection*{Proof}

Consider any $\boldsymbol{\lambda}^{(1)},\boldsymbol{\lambda}^{(2)}\in\Lambda$.
Let $\boldsymbol{\beta}_{j}=\hat{\boldsymbol{\theta}}_{\lambda^{(2)},j}-\hat{\boldsymbol{\theta}}_{\lambda^{(1)},j}$
for all $j=1,...,J$.

Define 
\[
\boldsymbol{\hat{m}}(\boldsymbol{\lambda})=\arg\min_{\boldsymbol{m}}\frac{1}{2}\left\Vert y-\sum_{j=1}^{J}g_{j}(\cdot|\hat{\boldsymbol{\theta}}_{\lambda^{(1)},j}+m_{j}\boldsymbol{\beta}_{j})\right\Vert _{T}^{2}+\sum_{j=1}^{J}\lambda_{j}\left(P_{j}(\hat{\boldsymbol{\theta}}_{\lambda^{(1)},j}+m_{j}\boldsymbol{\beta}_{j})+\frac{w}{2}\|\hat{\boldsymbol{\theta}}_{\lambda^{(1)},j}+m_{j}\boldsymbol{\beta}_{j}\|_{2}^{2}\right)
\]


By definition, we know that $\boldsymbol{\hat{m}}(\boldsymbol{\lambda}^{(2)})=\boldsymbol{1}$
and $\boldsymbol{\hat{m}}(\boldsymbol{\lambda^{(1)}})=\boldsymbol{0}$.

\textbf{1. We calculate $\nabla_{\lambda}\boldsymbol{\hat{m}_{k}(\lambda)}$
using the implicit differentiation trick.}

By the KKT conditions, we have for all $j=1:J$ 
\[
\left.\frac{\partial}{\partial m_{j}}\left(\frac{1}{2}\left\Vert y-\sum_{j=1}^{J}g_{j}(\cdot|\hat{\boldsymbol{\theta}}_{\lambda^{(1)},j}+m_{j}\boldsymbol{\beta}_{j})\right\Vert _{T}^{2}+\lambda_{j}P_{j}(\hat{\boldsymbol{\theta}}_{\lambda^{(1)},j}+m_{j}\boldsymbol{\beta}_{j})\right)+\lambda_{j}w\langle\boldsymbol{\beta}_{j},\hat{\boldsymbol{\theta}}_{\lambda^{(1)},j}+m_{j}\boldsymbol{\beta}_{j}\rangle\right|_{\boldsymbol{m}=\boldsymbol{\hat{m}}(\boldsymbol{\lambda})}=0
\]


Now we implicitly differentiate with respect to $\lambda_{\ell}$
for $\ell=1,2,...,J$

\[
\frac{\partial}{\partial\lambda_{\ell}}\left\{ \left.\left[\frac{\partial}{\partial m_{j}}\left(\frac{1}{2}\left\Vert y-\sum_{j=1}^{J}g_{j}(\cdot|\hat{\boldsymbol{\theta}}_{\lambda^{(1)},j}+m_{j}\boldsymbol{\beta}_{j})\right\Vert _{T}^{2}+\lambda_{j}P_{j}(\hat{\boldsymbol{\theta}}_{\lambda^{(1)},j}+m_{j}\boldsymbol{\beta}_{j})\right)+\lambda_{j}w\langle\boldsymbol{\beta}_{j},\hat{\boldsymbol{\theta}}_{\lambda^{(1)},j}+m_{j}\boldsymbol{\beta}_{j}\rangle\right]\right|_{\boldsymbol{m}=\hat{\boldsymbol{m}}(\boldsymbol{\lambda})}\right\} =0
\]


By the product rule and chain rule, we have 

\[
\left.\left\{ \left[\sum_{k=1}^{J}\left[\frac{\partial^{2}}{\partial m_{k}\partial m_{j}}\left(\frac{1}{2}\left\Vert y-\sum_{j=1}^{J}g_{j}(\cdot|\hat{\boldsymbol{\theta}}_{\lambda^{(1)},j}+m_{j}\boldsymbol{\beta}_{j})\right\Vert _{T}^{2}+1[k=j]\lambda_{j}P_{j}(\hat{\boldsymbol{\theta}}_{\lambda^{(1)},j}+m_{j}\boldsymbol{\beta}_{j})\right)+1[k=j]\lambda_{j}w\|\boldsymbol{\beta}_{j}\|_{2}^{2}\right]\frac{\partial}{\partial\lambda_{\ell}}\hat{m}_{k}(\boldsymbol{\lambda})\right]+1[j=\ell]\left(\frac{\partial}{\partial m}_{\ell}P_{\ell}(\hat{\boldsymbol{\theta}}_{\lambda^{(1)},\ell}+m_{\ell}\boldsymbol{\beta}_{\ell})+w\langle\boldsymbol{\beta}_{\ell},\hat{\boldsymbol{\theta}}_{\lambda^{(1)},\ell}+m_{\ell}\boldsymbol{\beta}_{\ell}\rangle\right)\right\} \right|_{\boldsymbol{m}=\hat{\boldsymbol{m}}(\boldsymbol{\lambda})}=0
\]


Define the following matrices 
\begin{eqnarray*}
S:S_{jk} & = & \left.\frac{\partial^{2}}{\partial m_{k}\partial m_{j}}\frac{1}{2}\|y-\sum_{j=1}^{J}g_{j}(\cdot|\hat{\boldsymbol{\theta}}_{\lambda^{(1)},j}+m_{j}\boldsymbol{\beta}_{j})\|_{T}^{2}\right|_{\boldsymbol{m}=\hat{\boldsymbol{m}}(\boldsymbol{\lambda})}
\end{eqnarray*}


\[
D_{1}=\left.diag\left(\frac{\partial^{2}}{\partial m_{j}^{2}}\lambda_{j}P_{j}(\hat{\boldsymbol{\theta}}_{\lambda^{(1)},j}+m_{j}\boldsymbol{\beta}_{j})\right)\right|_{\boldsymbol{m}=\hat{\boldsymbol{m}}(\boldsymbol{\lambda})}
\]


\[
D_{2}=diag\left(\lambda_{j}w\|\boldsymbol{\beta}_{j}\|_{2}^{2}\right)
\]


\[
D_{3}=\left.diag\left(\frac{\partial}{\partial m}_{\ell}P_{\ell}(\hat{\boldsymbol{\theta}}_{\lambda^{(1)},\ell}+m_{\ell}\boldsymbol{\beta}_{\ell})+w\langle\boldsymbol{\beta}_{\ell},\hat{\boldsymbol{\theta}}_{\lambda^{(1)},\ell}+m_{\ell}\boldsymbol{\beta}_{\ell}\rangle\right)\right|_{\boldsymbol{m}=\hat{\boldsymbol{m}}(\boldsymbol{\lambda})}
\]


\[
M=\left(\begin{array}{cccc}
\nabla_{\lambda}\hat{m}_{1}(\lambda) & \nabla_{\lambda}\hat{m}_{2}(\lambda) & ... & \nabla_{\lambda}\hat{m}_{J}(\lambda)\end{array}\right)
\]


We can then combine all the equations into the following system of
equations:
\[
M=-D_{3}\left(S+D_{1}+D_{2}\right)^{-1}
\]


$S$ is a PSD matrix since the sum of convex functions is convex (so
sum of $g_{j}$ is convex) and the composition of convex functions
is convex (so the composition of the mean squared error and the sum
of $g_{j}$ is convex).

$D_{1}$ is a PSD matrix since the penalty functions are convex.

\textbf{2. We bound every diagonal element in $D_{3}$}:

By assumption, we know for every $k=1,...,J$ 
\begin{equation}
\left|\frac{\partial}{\partial m}_{k}P_{k}(\hat{\boldsymbol{\theta}}_{\lambda^{(1)},k}+m_{k}\boldsymbol{\beta}_{k})\right|\le K\|\boldsymbol{\beta}_{k}\|
\end{equation}


Also, 
\begin{equation}
\left|w\langle\boldsymbol{\beta}_{k},\hat{\boldsymbol{\theta}}_{\lambda^{(1)},k}+\hat{m}_{k}(\boldsymbol{\lambda})\boldsymbol{\beta}_{k}\rangle\right|\le w\|\boldsymbol{\beta}_{k}\|\|\hat{\boldsymbol{\theta}}_{\lambda^{(1)},k}+\hat{m}_{k}(\boldsymbol{\lambda})\boldsymbol{\beta}_{k}\|
\end{equation}


To bound $\|\hat{\boldsymbol{\theta}}_{\lambda^{(1)},k}+\hat{m}_{k}(\boldsymbol{\lambda})\boldsymbol{\beta}_{k}\|$,
we use the basic inequality for $\hat{m}_{k}(\boldsymbol{\lambda})$:
\begin{eqnarray*}
\frac{\lambda_{k}w}{2}\|\hat{\boldsymbol{\theta}}_{\lambda^{(1)},k}+\hat{m}_{k}(\boldsymbol{\lambda})\boldsymbol{\beta}_{k}\|^{2} & \le & \frac{1}{2}\|y-\sum_{j=1}^{J}g(\cdot|\hat{\boldsymbol{\theta}}_{\lambda^{(1)},j})\|_{T}^{2}+\sum_{j=1}^{J}\lambda_{j}\left(P_{j}(\hat{\boldsymbol{\theta}}_{\lambda^{(1)},j})+\frac{w}{2}\|\hat{\boldsymbol{\theta}}_{\lambda^{(1)},j}\|_{2}^{2}\right)\\
 & = & \frac{1}{2}\|y-\sum_{j=1}^{J}g(\cdot|\hat{\boldsymbol{\theta}}_{\lambda^{(1)},j})\|_{T}^{2}+\sum_{j=1}^{J}\lambda_{j}^{(1)}\left(P_{j}(\hat{\boldsymbol{\theta}}_{\lambda^{(1)},j})+\frac{w}{2}\|\hat{\boldsymbol{\theta}}_{\lambda^{(1)},j}\|_{2}^{2}\right)+\sum_{j=1}^{J}(\lambda_{j}-\lambda_{j}^{(1)})\left(P_{j}(\hat{\boldsymbol{\theta}}_{\lambda^{(1)},j})+\frac{w}{2}\|\hat{\boldsymbol{\theta}}_{\lambda^{(1)},j}\|_{2}^{2}\right)\\
 & \le & C+J\lambda_{max}\max_{j=1:J}\left(P_{j}(\hat{\boldsymbol{\theta}}_{\lambda^{(1)},j})+\frac{w}{2}\|\hat{\boldsymbol{\theta}}_{\lambda^{(1)},j}\|_{2}^{2}\right)
\end{eqnarray*}


To bound the term $\max_{j=1:J}\left(P_{j}(\hat{\boldsymbol{\theta}}_{\lambda^{(1)},j})+\frac{w}{2}\|\hat{\boldsymbol{\theta}}_{\lambda^{(1)},j}\|_{2}^{2}\right)$,
we use the basic inequality for $\hat{\boldsymbol{\theta}}_{\lambda^{(1)}}$:
\begin{eqnarray*}
\sum_{j=1}^{J}\lambda_{j}^{(1)}\left(P_{j}(\hat{\boldsymbol{\theta}}_{\lambda^{(1)},j})+\frac{w}{2}\|\hat{\boldsymbol{\theta}}_{\lambda^{(1)},j}\|_{2}^{2}\right) & \le & \frac{1}{2}\|y-\sum_{j=1}^{J}g(\cdot|\hat{\boldsymbol{\theta}}_{j}^{*})\|_{T}^{2}+\sum_{j=1}^{J}\lambda_{j}^{(1)}\left(P_{j}(\hat{\boldsymbol{\theta}}_{j}^{*})+\frac{w}{2}\|\hat{\boldsymbol{\theta}}_{j}^{*}\|_{2}^{2}\right)\\
 & \le & C
\end{eqnarray*}


Since 
\[
\lambda_{min}\left(\max_{j=1:J}P_{j}(\hat{\boldsymbol{\theta}}_{\lambda^{(1)},j})+\frac{w}{2}\|\hat{\boldsymbol{\theta}}_{\lambda^{(1)},j}\|_{2}^{2}\right)\le\sum_{j=1}^{J}\lambda_{j}^{(1)}\left(P_{j}(\hat{\boldsymbol{\theta}}_{\lambda^{(1)},j})+\frac{w}{2}\|\hat{\boldsymbol{\theta}}_{\lambda^{(1)},j}\|_{2}^{2}\right)
\]


then we have that
\[
\max_{j=1:J}P_{j}(\hat{\boldsymbol{\theta}}_{\lambda^{(1)},j})+\frac{w}{2}\|\hat{\boldsymbol{\theta}}_{\lambda^{(1)},j}\|_{2}^{2}\le\frac{C}{\lambda_{min}}
\]


Therefore for all $k=1,...,J$
\[
\frac{\lambda_{k}w}{2}\|\hat{\boldsymbol{\theta}}_{\lambda^{(1)},k}+m_{\ell}\boldsymbol{\beta}_{k}\|^{2}\le\left(1+\frac{J\lambda_{max}}{\lambda_{min}}\right)C
\]


Rearranging, we get
\begin{equation}
\|\hat{\boldsymbol{\theta}}_{\lambda^{(1)},k}+m_{\ell}\boldsymbol{\beta}_{k}\|\le\sqrt{\left(1+\frac{J\lambda_{max}}{\lambda_{min}}\right)\frac{2C}{\lambda_{min}w}}
\end{equation}


Therefore 
\[
\left|\frac{\partial}{\partial m}_{k}P_{k}(\hat{\boldsymbol{\theta}}_{\lambda^{(1)},k}+m_{k}\boldsymbol{\beta}_{k})+w\langle\boldsymbol{\beta}_{k},\hat{\boldsymbol{\theta}}_{\lambda^{(1)},k}+m_{k}\boldsymbol{\beta}_{k}\rangle\right|_{\boldsymbol{m}=\hat{\boldsymbol{m}}(\boldsymbol{\lambda})}\le K\|\boldsymbol{\beta}_{k}\|+w\|\boldsymbol{\beta}_{k}\|\sqrt{\left(1+\frac{J\lambda_{max}}{\lambda_{min}}\right)\frac{2C}{\lambda_{min}w}}
\]


Let 
\[
D_{3,upper}=\left(K+w\sqrt{\left(1+\frac{J\lambda_{max}}{\lambda_{min}}\right)\frac{2C}{\lambda_{min}w}}\right)diag\left(\|\boldsymbol{\beta}_{k}\|\right)
\]


We know that $D_{3,upper}\succeq D_{3}$.

\textbf{3. We bound the norm of $\nabla_{\lambda}\hat{m}_{k}(\lambda)$
for all $k=1,...,J$.} 
\begin{eqnarray}
\|\nabla_{\lambda}\hat{m}_{k}(\lambda)\| & = & \|Me_{k}\|\nonumber \\
 & = & \|D_{3}\left(S+D_{1}+D_{2}\right)^{-1}e_{k}\|\nonumber \\
 & \le & \|D_{3,upper}\left(S+D_{1}+D_{2}\right)^{-1}e_{k}\|\nonumber \\
 & \le & \left(K+w\sqrt{\left(1+\frac{J\lambda_{max}}{\lambda_{min}}\right)\frac{2C}{\lambda_{min}w}}\right)\max_{\ell}\|\boldsymbol{\beta}_{\ell}\|\left\Vert \left(S+D_{1}+D_{2}\right)^{-1}e_{k}\right\Vert \nonumber \\
 & \le & \left(K+w\sqrt{\left(1+\frac{J\lambda_{max}}{\lambda_{min}}\right)\frac{2C}{\lambda_{min}w}}\right)\max_{\ell}\|\boldsymbol{\beta}_{\ell}\|\left\Vert D_{2}^{-1}e_{k}\right\Vert 
\end{eqnarray}


The last line follows from the matrix inverse lemma: Since $S+D_{1}$
is a PSD matrix, then 
\[
\left\Vert \left(S+D_{1}+D_{2}\right)^{-1}e_{k}\right\Vert \le\left\Vert D_{2}^{-1}e_{k}\right\Vert 
\]


Now let 
\[
\ell_{max}=\arg\max_{\ell}\|\boldsymbol{\beta}_{\ell}\|
\]


If we consider (8) for $k=\ell_{max}$, then 
\begin{eqnarray*}
\|\nabla_{\lambda}\hat{m}_{\ell_{max}}(\lambda)\| & \le & \left(K+w\sqrt{\left(1+\frac{J\lambda_{max}}{\lambda_{min}}\right)\frac{2C}{\lambda_{min}w}}\right)\|\boldsymbol{\beta}_{\ell_{max}}\|\left\Vert D_{2}^{-1}e_{\ell_{max}}\right\Vert \\
 & = & \left(K+w\sqrt{\left(1+\frac{J\lambda_{max}}{\lambda_{min}}\right)\frac{2C}{\lambda_{min}w}}\right)\|\boldsymbol{\beta}_{\ell_{max}}\|\lambda_{\ell_{max}}^{-1}w^{-1}\|\boldsymbol{\beta}_{\ell_{max}}\|_{2}^{-2}\\
 & = & \left(K+w\sqrt{\left(1+\frac{J\lambda_{max}}{\lambda_{min}}\right)\frac{2C}{\lambda_{min}w}}\right)\|\boldsymbol{\beta}_{\ell_{max}}\|^{-1}\lambda_{min}^{-1}w^{-1}
\end{eqnarray*}


\textbf{4. Apply the Mean Value Theorem}

Since the training criterion is smooth, then $\hat{m}_{\ell_{max}}(\lambda)$
is a continuous, differentiable function.

By the MVT, we have that there exists an $\alpha\in(0,1)$ such that
\begin{eqnarray*}
\left|\hat{m}_{\ell_{max}}(\boldsymbol{\lambda}^{(2)})-\hat{m}_{\ell_{max}}(\boldsymbol{\lambda}^{(1)})\right| & = & \left|\left\langle \boldsymbol{\lambda}^{(2)}-\boldsymbol{\lambda}^{(1)},\nabla_{\lambda}\hat{m}_{\ell_{max}}(\boldsymbol{\lambda})\right\rangle _{\boldsymbol{\lambda}=\alpha\boldsymbol{\lambda}^{(1)}+(1-\alpha)\boldsymbol{\lambda}^{(2)}}\right|\\
 & \le & \left\Vert \boldsymbol{\lambda}^{(2)}-\boldsymbol{\lambda}^{(1)}\right\Vert \left|K+w\sqrt{\left(1+\frac{J\lambda_{max}}{\lambda_{min}}\right)\frac{2C}{\lambda_{min}w}}\right|\lambda_{min}^{-1}w^{-1}\|\boldsymbol{\beta}_{\ell_{max}}\|^{-1}
\end{eqnarray*}


We know that $\hat{m}_{k}(\boldsymbol{\lambda}^{(2)})-\hat{m}_{k}(\boldsymbol{\lambda}^{(1)})=\boldsymbol{1}$
for all $k=1,..,J$. Rearranging the inequality above, we get
\[
\max_{k}\|\boldsymbol{\theta}_{\lambda^{(1)},k}-\boldsymbol{\theta}_{\lambda^{(2)},k}\|=\|\boldsymbol{\beta}_{\ell_{max}}\|\le\left\Vert \boldsymbol{\lambda}^{(2)}-\boldsymbol{\lambda}^{(1)}\right\Vert \left|K+w\sqrt{\left(1+\frac{J\lambda_{max}}{\lambda_{min}}\right)\frac{2C}{\lambda_{min}w}}\right|\lambda_{min}^{-1}w^{-1}
\]



\section{Nonsmooth Penalties}

Suppose we are dealing with parametric regression problems from Section
1 or 2. We will suppose all the same assumptions, except those that
concern the smoothness of the penalties. 

Assumption modification (1): Suppose there is some $K>0$ such that
for all $j=1,...,J$ and any $\boldsymbol{\theta,\beta}$ such that

\[
\left|\frac{\partial}{\partial m}P_{j}\left(\boldsymbol{\theta}+m\boldsymbol{\beta}\right)\right|\le K\|\boldsymbol{\beta}\|_{2}\mbox{ if }\frac{\partial}{\partial m}P_{j}\left(\boldsymbol{\theta}+m\boldsymbol{\beta}\right)\mbox{exists}
\]


Assumption modification (2): The non-smooth training criterion satisfy
the Conditions 1, 2, and 3 from the Hillclimbing paper. Denote the
differentiable space of $L_{T}(\cdot,\boldsymbol{\lambda})$ at any
point $\boldsymbol{\theta}$ as 
\[
\Omega^{L_{T}(\cdot,\boldsymbol{\lambda})}\left(\boldsymbol{\theta}\right)
\]


For every $\boldsymbol{\lambda}\in\Lambda_{smooth}$, we have 

\textbf{Cond 1:} The differentiable space of the training criterion
at $\hat{\boldsymbol{\theta}}(\boldsymbol{\lambda})$, denoted $\Omega^{L_{T}(\cdot,\boldsymbol{\lambda})}\left(\hat{\boldsymbol{\theta}}(\boldsymbol{\lambda})\right)$,
is a local optimality space.

\textbf{Cond 2:} The training criterion $L_{T}(\cdot,\cdot)$ restricted
to $\Omega^{L_{T}(\cdot,\cdot)}\left(\hat{\boldsymbol{\theta}}(\boldsymbol{\lambda}),\boldsymbol{\lambda}\right)$
is twice continuously differentiable within some ball centered $\boldsymbol{\lambda}$.
Let ``ball of differentiability'' be denoted $B(\boldsymbol{\lambda})$.

\textbf{Cond 3:} There is an orthonormal basis $U$ of the differentiable
space directions such that the Hessian of the training criterion (taken
along directions $U$) is invertible.

Suppose that 
\[
\mu(\Lambda_{smooth}^{C})=0
\]


Under these non-smooth conditions, the same Lipschitz condition will
hold.


\subsubsection*{Proof}

Now define

\[
L_{nonsmooth}=\left\{ \mbox{line that passes through }\boldsymbol{\lambda}_{1},\boldsymbol{\lambda}_{2}:\boldsymbol{\lambda}_{1},\boldsymbol{\lambda}_{2}\in\Lambda_{smooth}^{C}\right\} 
\]


\textbf{Unproven Claim:} Since $\mu(\Lambda_{smooth}^{C})=0$, then
$\mu(L_{nonsmooth})=0$. I don't know how to prove this clain, but
it seems true.

Now denote the line segment between $\boldsymbol{\lambda^{(1)}},\boldsymbol{\lambda^{(2)}}$
as 
\[
\mathcal{L}(\boldsymbol{\lambda^{(1)}},\boldsymbol{\lambda^{(2)}})=\left\{ \alpha\boldsymbol{\lambda^{(1)}}+(1-\alpha)\boldsymbol{\lambda^{(2)}}:\alpha\in[0,1]\right\} 
\]


The set
\[
H=\left\{ (\boldsymbol{\lambda^{(1)}},\boldsymbol{\lambda^{(2)}}):\left\Vert \mathcal{L}(\boldsymbol{\lambda^{(1)}},\boldsymbol{\lambda^{(2)}})\cap\Lambda_{smooth}^{C}\right\Vert >0\right\} 
\]
 has measure $\mu(H)=0$ since $H\subseteq L_{nonsmooth}$.

Now consider any line segment $\mathcal{L}(\boldsymbol{\lambda^{(1)}},\boldsymbol{\lambda^{(2)}})$
not in $H^{C}$. We want to show that there is a set of points $\{\boldsymbol{\ell}^{(i)}\}$
along $\mathcal{L}(\boldsymbol{\lambda^{(1)}},\boldsymbol{\lambda^{(2)}})$
such that the ``balls of differentiability'' $B(\boldsymbol{\ell}^{(i)})$
cover the entire line segment. We will define a function to measure
this uncovered distance: For a given set of points $\{\boldsymbol{\ell}^{(i)}\}\subset\mathcal{L}(\boldsymbol{\lambda^{(1)}},\boldsymbol{\lambda^{(2)}})$,
let $d(\{\boldsymbol{\ell}^{(i)}\})$ denote the covered distance
of $\mathcal{L}(\boldsymbol{\lambda^{(1)}},\boldsymbol{\lambda^{(2)}})$
by the union of their differentiable spaces:
\[
d_{\boldsymbol{\lambda^{(1)}},\boldsymbol{\lambda^{(2)}}}(\{\boldsymbol{\ell}^{(i)}\})=\left\Vert \left[\cup_{i}B(\boldsymbol{\ell}^{(i)})\right]\cap\mathcal{L}(\boldsymbol{\lambda^{(1)}},\boldsymbol{\lambda^{(2)}})\right\Vert 
\]


\textbf{Claim:} For all $(\boldsymbol{\lambda^{(1)}},\boldsymbol{\lambda^{(2)}})\in H^{C}$,
there is a set of points $\{\boldsymbol{\ell}^{(i)}\}\subseteq\mathcal{L}(\boldsymbol{\lambda^{(1)}},\boldsymbol{\lambda^{(2)}})$
such that their ``balls of differentiability'' completely cover
$\mathcal{L}(\boldsymbol{\lambda^{(1)}},\boldsymbol{\lambda^{(2)}})$:
\[
\max d_{\boldsymbol{\lambda^{(1)}},\boldsymbol{\lambda^{(2)}}}\left(\left\{ \boldsymbol{\ell}^{(i)}\right\} \right)=\|\mathcal{L}(\boldsymbol{\lambda^{(1)}},\boldsymbol{\lambda^{(2)}})\|
\]


\textbf{Proof of Claim:}

For contradiction, suppose that no set of points can cover the line
segment. For notational convenience, let us write 
\[
\bar{\boldsymbol{\ell}}_{max}=\arg\max_{\{\ell^{(i)}\}}d_{\boldsymbol{\lambda^{(1)}},\boldsymbol{\lambda^{(2)}}}(\{\boldsymbol{\ell}^{(i)}\})
\]


So 
\[
d_{\boldsymbol{\lambda^{(1)}},\boldsymbol{\lambda^{(2)}}}\left(\bar{\boldsymbol{\ell}}_{max}\right)<\|\mathcal{L}(\boldsymbol{\lambda^{(1)}},\boldsymbol{\lambda^{(2)}})\|
\]


Let $\mathcal{L}_{U}$ be the set of points left uncovered:

\[
\mathcal{L}_{uncovered}=\mathcal{L}(\boldsymbol{\lambda^{(1)}},\boldsymbol{\lambda^{(2)}})\backslash\left[\cup_{\ell\in\bar{\boldsymbol{\ell}}_{max}}B(\boldsymbol{\ell})\right]
\]


So 
\[
\left\Vert \mathcal{L}(\boldsymbol{\lambda^{(1)}},\boldsymbol{\lambda^{(2)}})\cap U\right\Vert <\|\mathcal{L}(\boldsymbol{\lambda^{(1)}},\boldsymbol{\lambda^{(2)}})\|
\]


There are two cases:

(1) $\mathcal{L}_{uncovered}\subseteq\Lambda_{smooth}^{C}$. Then
$\left\Vert \mathcal{L}(\boldsymbol{\lambda^{(1)}},\boldsymbol{\lambda^{(2)}})\cap\Lambda_{smooth}^{C}\right\Vert \ge\left\Vert \mathcal{L}(\boldsymbol{\lambda^{(1)}},\boldsymbol{\lambda^{(2)}})\cap\mathcal{L}_{uncovered}\right\Vert >0$.
This is clearly impossible since $(\boldsymbol{\lambda^{(1)}},\boldsymbol{\lambda^{(2)}})\in H^{C}$. 

(2) There exists a point $\boldsymbol{p}\in\mathcal{L}_{uncovered}\backslash\Lambda_{smooth}^{C}$.
Since $\boldsymbol{p}\in\Lambda_{smooth}$, then by Condition 2, then
the neighborhood $B(\boldsymbol{p})$ is non-empty. 
\[
\left\Vert B(\boldsymbol{p})\cap\mathcal{L}(\boldsymbol{\lambda^{(1)}},\boldsymbol{\lambda^{(2)}})\right\Vert >0
\]


This implies that
\[
d_{\boldsymbol{\lambda^{(1)}},\boldsymbol{\lambda^{(2)}}}\left(\bar{\boldsymbol{\ell}}_{max}\right)<d_{\boldsymbol{\lambda^{(1)}},\boldsymbol{\lambda^{(2)}}}\left(\bar{\boldsymbol{\ell}}_{max}\cup\{\boldsymbol{p}\}\right)
\]


However contradicts the definition of $\bar{\boldsymbol{\ell}}_{max}$
that it maximizes the covered distance.

\textbf{End of Proof}

From the claim above, let's consider any $(\boldsymbol{\lambda^{(1)}},\boldsymbol{\lambda^{(2)}})\in H^{C}$.
Let 
\[
\bar{\boldsymbol{\ell}}_{max}=\arg\max_{\left\{ \boldsymbol{\ell}^{(i)}\right\} }d_{\boldsymbol{\lambda^{(1)}},\boldsymbol{\lambda^{(2)}}}\left(\left\{ \boldsymbol{\ell}^{(i)}\right\} \right)
\]


Then define the intersections of the edges of the ``balls of differentiability''
with the line segment $\mathcal{L}(\boldsymbol{\lambda^{(1)}},\boldsymbol{\lambda^{(2)}})$.
\[
P=\left\{ \mbox{The points at the edge of }B(\boldsymbol{\ell})\mbox{ that intersect with }\mathcal{L}(\boldsymbol{\lambda^{(1)}},\boldsymbol{\lambda^{(2)}}):\boldsymbol{\ell}\in\bar{\boldsymbol{\ell}}_{max}\right\} \cup\{\boldsymbol{\lambda}^{(1)},\boldsymbol{\lambda}^{(2)}\}
\]


Since every point can be expressed as $\alpha_{p^{(i)}}\boldsymbol{\lambda^{(1)}}+(1-\alpha_{p^{(i)}})\boldsymbol{\lambda^{(2)}}$
for some $\alpha_{p^{(i)}}\in[0,1]$, we can order these points $\{\boldsymbol{p}^{(i)}\}$
by increasing $\alpha_{p^{(i)}}$.

By definition of $P$ and the Claim, the differentiable space of the
training criterion over $\left(\boldsymbol{p^{(i)},p^{(i+1)}}\right)$
must be constant. 

We can apply the smoothness result in Section 1 or 2 over every interval
$\left(\boldsymbol{p^{(i)},p^{(i+1)}}\right)$ since we can come up
with an equivalent definition for $\hat{\boldsymbol{\theta}}(\boldsymbol{\lambda})$:
There is an orthonormal matrix $U^{(i)}$ such that for all $\boldsymbol{\lambda}\in\left(\boldsymbol{p^{(i)},p^{(i+1)}}\right)$
\[
\hat{\boldsymbol{\theta}}_{\lambda}=U^{(i)}\hat{\boldsymbol{\beta}}_{\lambda}
\]
\[
\hat{\boldsymbol{\beta}}_{\lambda}=\arg\min_{\beta}L_{T}(U^{(i)}\boldsymbol{\beta},\boldsymbol{\lambda})
\]


where the training criterion is smooth over $\left(\boldsymbol{p^{(i)},p^{(i+1)}}\right)$
wrt to the directional derivatives along the columns of $U^{(i)}$.

For example, in the case of Section 1, we would instead consider regression
problems of the form 
\begin{eqnarray*}
\hat{\boldsymbol{\beta}}_{\lambda} & = & \arg\min_{\boldsymbol{\beta}}\frac{1}{2}\|y-g(\cdot|U\boldsymbol{\beta})\|_{T}^{2}+\sum_{j=1}^{J}\lambda_{j}\left(P_{j}(U\boldsymbol{\beta})+\frac{w}{2}\|U\boldsymbol{\beta}\|_{2}^{2}\right)\\
 & = & \arg\min_{\boldsymbol{\beta}}\frac{1}{2}\|y-g(\cdot|U\boldsymbol{\beta})\|_{T}^{2}+\sum_{j=1}^{J}\lambda_{j}\left(P_{j}(U\boldsymbol{\beta})+\frac{w}{2}\|\boldsymbol{\beta}\|_{2}^{2}\right)
\end{eqnarray*}


The proof from Sections 1 and 2 would need to be modified to take
directional derivatives along the columns of $U$. Applying the Section
1 or 2 results to each interval $\left(\boldsymbol{p^{(i)},p^{(i+1)}}\right)$
, we would get Lipschitz conditions of the form 
\[
\|\boldsymbol{\hat{\beta}}_{p^{(i)}}-\boldsymbol{\hat{\beta}}_{p^{(i+1)}}\|_{2}\le c\|\boldsymbol{p^{(i)}}-\boldsymbol{p^{(i+1)}}\|_{2}
\]


where $c$ is some constant.

Finally, we can sum up these inequalities to show smoothness of $\hat{\boldsymbol{\theta}}_{\lambda}$.
By the triangle inequality,
\begin{eqnarray*}
\|\boldsymbol{\hat{\theta}_{\lambda^{(1)}}}-\boldsymbol{\hat{\theta}_{\lambda^{(2)}}}\|_{2} & \le & \sum_{i=1}\|\boldsymbol{\hat{\theta}}_{p^{(i)}}-\boldsymbol{\hat{\theta}}_{p^{(i+1)}}\|_{2}\\
 & = & \sum_{i=1}\|\boldsymbol{\hat{\beta}}_{p^{(i)}}-\boldsymbol{\hat{\beta}}_{p^{(i+1)}}\|_{2}\\
 & \le & \sum_{i=1}c\|\boldsymbol{p^{(i)}}-\boldsymbol{p^{(i+1)}}\|_{2}\\
 & = & c\|\boldsymbol{\lambda}^{(1)}-\boldsymbol{\lambda}^{(2)}\|_{2}
\end{eqnarray*}



\section{Example}


\subsection{Penalties that satisfy the conditions}

\textbf{Ridge:}

The perturbation isn't necessary if there is already a ridge penalty
in the original penalized regression problem. Just set the penalties
$P_{j}(\boldsymbol{\theta})\equiv0$ and fix $w=2$.

\textbf{Lasso:}

\begin{eqnarray*}
\frac{\partial}{\partial m}\|\theta+m\beta\|_{1} & = & \langle sgn(\theta+m\beta),\beta\rangle\\
 & \le & \|sgn(\theta+m\beta)\|_{2}\|\beta\|_{2}\\
 & \le & p\|\beta\|_{2}
\end{eqnarray*}


\textbf{Generalized Lasso:} let $G$ be the maximum eigenvalue of
$D$.
\begin{eqnarray*}
\frac{\partial}{\partial m}\|D(\theta+m\beta)\|_{1} & = & \langle sgn(D(\theta+m\beta)),D\beta\rangle\\
 & \le & \|sgn(D(\theta+m\beta))\|_{2}\|D\beta\|_{2}\\
 & \le & pG\|\beta\|_{2}
\end{eqnarray*}


\textbf{Group Lasso:}

\begin{eqnarray*}
\frac{\partial}{\partial m}\|\theta+m\beta\|_{2} & = & \langle\frac{\theta+m\beta}{\|\theta+m\beta\|_{2}},\beta\rangle\\
 & \le & \|\beta\|_{2}
\end{eqnarray*}



\subsection{Sobolev}

Given a function $h$, the Sobolev penalty for $h$ is

\[
P(h)=\int(h^{(r)}(x))^{2}dx
\]


The Sobolev penalty is used in nonparametric regression models, but
such nonparametric regression models can be re-expressed in parametric
form. We will use this to understand the smoothness of models fitted
in this manner.

Consider the class of smoothing splines 
\[
\left\{ \hat{g}(\cdot|\lambda)=\arg\min_{g\in\mathcal{G}}\frac{1}{2}\|y-\sum_{j=1}^{J}g_{j}(x_{j})\|_{T}^{2}+\sum_{j=1}^{J}\lambda_{j}P(g_{j}):\lambda\in\Lambda\right\} 
\]


Each function $\hat{g}_{j}(\cdot|\lambda)$ is a spline that can be
expressed as the weighted sum of $B$ normalized B-splines of degree
$r+1$ for a given set of knots:
\[
\hat{g}_{j}(x|\lambda)=\sum_{i=1}^{B}\theta_{i}N_{j,i}(x)
\]


Note that the normalized B-splines have the property that they sum
up to one at all points within the boundary of the knots. Also recall
that B-splines are non-negative.

Therefore we can re-express the class of smoothing splines as a set
of function parameters
\[
\left\{ \hat{\boldsymbol{\theta}}_{\lambda}=\arg\min_{\theta}\frac{1}{2}\|y-\sum_{j=1}^{J}N_{T,j}\boldsymbol{\theta}_{j}\|^{2}++\sum_{j=1}^{J}\lambda_{j}P_{j}(\boldsymbol{\theta}_{j}):\lambda\in\Lambda\right\} 
\]
where $N_{T,j}$ is the normalized B-spline basis for the given set
of knots evaluated at the observed $x_{j}$ in the training set. $P_{j}(\boldsymbol{\theta_{j}})$
is the Sobolev penalty and can be written as $\boldsymbol{\theta}_{j}^{T}\Omega_{j}\boldsymbol{\theta}_{j}$
for an appropriate penalty matrix $\Omega_{j}$. We will not need
to express anything in terms of $\Omega_{j}$ so the penalty will
be just written as $P_{j}(\boldsymbol{\theta}_{j})$.

Instead of considering the original smoothing spline problem with
the roughness penalty, we will add a ridge penalty on the function
parameters
\[
\left\{ \hat{\boldsymbol{\theta}}_{\lambda}=\arg\min_{\theta}\frac{1}{2}\|y-\sum_{j=1}^{J}N_{T,j}\boldsymbol{\theta}_{j}\|^{2}++\sum_{j=1}^{J}\lambda_{j}\left(P_{j}(\boldsymbol{\theta}_{j})+\frac{w}{2}\|\boldsymbol{\theta}_{j}\|_{2}^{2}\right):\lambda\in\Lambda\right\} 
\]


Let

\[
K=\frac{1}{\lambda_{min}}\left(B+\lambda_{max}\sqrt{\frac{w}{\lambda_{min}}}\right)\sqrt{\left(1+\frac{J\lambda_{max}}{\lambda_{min}}\right)2C}
\]


By Section 2, for any $\boldsymbol{\lambda^{(1)},\lambda^{(2)}}\in\Lambda$
we have for all $j=1,...,J$
\[
\left\Vert \boldsymbol{\theta}_{\lambda^{(1)},j}-\boldsymbol{\theta}_{\lambda^{(2)},j}\right\Vert \le\left\Vert \boldsymbol{\lambda}^{(2)}-\boldsymbol{\lambda}^{(1)}\right\Vert \left(K+w\sqrt{\frac{2C}{\lambda_{min}w}\left(1+\frac{J\lambda_{max}}{\lambda_{min}}\right)}\right)\lambda_{min}^{-1}w^{-1}
\]


Moreover,

\[
\left\Vert \sum_{j=1}^{J}\hat{g}_{j}(x_{j}|\lambda^{(1)})-\hat{g}_{j}(x_{j}|\lambda^{(2)})\right\Vert _{\infty}\le\left\Vert \boldsymbol{\lambda}^{(2)}-\boldsymbol{\lambda}^{(1)}\right\Vert J\sqrt{B}\left(K+w\sqrt{\frac{2C}{\lambda_{min}w}\left(1+\frac{J\lambda_{max}}{\lambda_{min}}\right)}\right)\lambda_{min}^{-1}w^{-1}
\]



\subsubsection*{Proof}

Consider any $\boldsymbol{\lambda}^{(1)},\boldsymbol{\lambda}^{(2)}\in\Lambda$.
Let $\boldsymbol{\beta}_{j}=\boldsymbol{\theta}_{\lambda^{(1)},j}-\boldsymbol{\theta}_{\lambda^{(2)},j}$
for all $j=1,...,J$. Define 
\[
\boldsymbol{\hat{m}}(\boldsymbol{\lambda})=\arg\min_{\boldsymbol{m}}\frac{1}{2}\|y-\sum_{j=1}^{J}N_{T,j}(\hat{\boldsymbol{\theta}}_{\lambda^{(1)},j}+m_{j}\boldsymbol{\beta}_{j})\|_{T}^{2}+\sum_{j=1}^{J}\lambda_{j}\left(P_{j}(\hat{\boldsymbol{\theta}}_{\lambda^{(1)},j}+m_{j}\boldsymbol{\beta}_{j})+\frac{w}{2}\|\hat{\boldsymbol{\theta}}_{\lambda^{(1)},j}+m_{j}\boldsymbol{\beta}_{j}\|_{2}^{2}\right)
\]


We are interested in finding the value $K$ that satisfies the condition
such that 
\[
\left|\left.\frac{\partial}{\partial m_{j}}P_{j}(\hat{\boldsymbol{\theta}}_{\lambda^{(1)},j}+m_{j}\boldsymbol{\beta}_{j})\right|_{m=\hat{m}_{j}(\lambda)}\right|\le K\|\boldsymbol{\beta}_{j}\|
\]


Once this is true, Section 2's results can be applied immediately.

\textbf{1. Determine $\left.\frac{\partial}{\partial m}P_{j}(\hat{\boldsymbol{\theta}}_{\lambda^{(1)},j}+m\boldsymbol{\beta}_{j})\right|_{m=\hat{m}_{j}(\lambda)}$}

By the KKT conditions, we have for all $k=1:J$ 
\[
\left.\frac{\partial}{\partial m_{k}}\left(\frac{1}{2}\left\Vert y-\sum_{j=1}^{J}N_{T,j}(\hat{\boldsymbol{\theta}}_{\lambda^{(1)},j}+m_{j}\boldsymbol{\beta}_{j})\right\Vert _{T}^{2}+\lambda_{k}P_{k}(\hat{\boldsymbol{\theta}}_{\lambda^{(1)},k}+m_{k}\boldsymbol{\beta}_{k})\right)+\lambda_{k}w\langle\boldsymbol{\beta}_{k},\boldsymbol{\hat{\theta}_{\lambda^{(1)},k}}+m_{k}\boldsymbol{\beta}_{k}\rangle\right|_{\boldsymbol{m}=\boldsymbol{\hat{m}}(\boldsymbol{\lambda})}=0
\]


Rearranging, we get
\begin{eqnarray*}
\lambda_{k}\left.\frac{\partial}{\partial m_{k}}P_{k}(\hat{\boldsymbol{\theta}}_{\lambda^{(1)},k}+m_{k}\boldsymbol{\beta}_{k})\right|_{m_{k}=\hat{m}_{k}(\lambda)} & = & \left\langle N_{T,k}\boldsymbol{\beta}_{k},y-\sum_{j=1}^{J}N_{T,j}(\hat{\boldsymbol{\theta}}_{\lambda^{(1)},j}+\hat{m}_{j}(\boldsymbol{\lambda})\boldsymbol{\beta}_{j})\right\rangle _{T}+\lambda_{k}w\langle\boldsymbol{\beta}_{k},\boldsymbol{\hat{\theta}_{\lambda^{(1)},k}}+\hat{m}_{k}(\boldsymbol{\lambda})\boldsymbol{\beta}_{k}\rangle
\end{eqnarray*}


\textbf{2. Bound $\left|\left.\frac{\partial}{\partial m_{j}}P_{j}(\hat{\boldsymbol{\theta}}_{\lambda^{(1)},j}+m_{j}\boldsymbol{\beta}_{j})\right|_{m_{j}=\hat{m}_{j}(\lambda)}\right|$}

By Cauchy Schwarz, 
\[
\left|\left.\frac{\partial}{\partial m_{k}}P_{k}(\hat{\boldsymbol{\theta}}_{\lambda^{(1)},k}+m_{k}\boldsymbol{\beta}_{k})\right|_{m_{k}=\hat{m}_{k}(\lambda)}\right|\le\frac{1}{\lambda_{min}}\left(\left(\|N_{T,k}\boldsymbol{\beta}_{k}\|\left\Vert y_{T}-\sum_{j=1}^{J}N_{T,j}(\hat{\boldsymbol{\theta}}_{\lambda^{(1)},j}+\hat{m}_{j}(\boldsymbol{\lambda})\boldsymbol{\beta}_{j})\right\Vert \right)+\lambda_{k}w\|\boldsymbol{\beta}_{k}\|\left\Vert \boldsymbol{\hat{\theta}_{\lambda^{(1)},k}}+\hat{m}_{k}(\boldsymbol{\lambda})\boldsymbol{\beta}_{k}\right\Vert \right)
\]


Note that the eigenvalue of $N_{T,k}$ is bounded by $B$ since the
maximum eigenvalue of a non-negative matrix is bounded by its maximum
row sum. In the case of $N_{T,k}$, since it is the values of normalized
B-splines, each row is at most the number of B-spline basis functions.
That is, we have that
\[
\|N_{T,k}\boldsymbol{\beta}_{k}\|\le B\|\boldsymbol{\beta}_{k}\|
\]


To bound the other terms, we use the definition of $\hat{\boldsymbol{m}}(\boldsymbol{\lambda})$:
\begin{eqnarray*}
\frac{1}{2}\left\Vert y-\sum_{j=1}^{J}N_{T,j}(\hat{\boldsymbol{\theta}}_{\lambda^{(1)},j}+\hat{m}_{j}(\boldsymbol{\lambda})\boldsymbol{\beta}_{j})\right\Vert _{T}^{2}+\sum_{j=1}^{J}\frac{\lambda_{j}w}{2}\|\hat{\boldsymbol{\theta}}_{\lambda^{(1)},j}+\hat{m}_{j}(\boldsymbol{\lambda})\boldsymbol{\beta}_{j}\|^{2} & \le & \frac{1}{2}\left\Vert y-\sum_{j=1}^{J}N_{T,j}\hat{\boldsymbol{\theta}}_{\lambda^{(1)},j}\right\Vert _{T}^{2}+\sum_{j=1}^{J}\lambda_{j}\left(P_{j}(\hat{\boldsymbol{\theta}}_{\lambda^{(1)},j})+\frac{w}{2}\|\hat{\boldsymbol{\theta}}_{\lambda^{(1)},j}\|_{2}^{2}\right)\\
 & = & \frac{1}{2}\left\Vert y-\sum_{j=1}^{J}N_{T,j}\hat{\boldsymbol{\theta}}_{\lambda^{(1)},j}\right\Vert _{T}^{2}+\sum_{j=1}^{J}\lambda_{j}^{(1)}\left(P_{j}(\hat{\boldsymbol{\theta}}_{\lambda^{(1)},j})+\frac{w}{2}\|\hat{\boldsymbol{\theta}}_{\lambda^{(1)},j}\|_{2}^{2}\right)+\sum_{j=1}^{J}\left(\lambda_{j}-\lambda_{j}^{(1)}\right)\left(P_{j}(\hat{\boldsymbol{\theta}}_{\lambda^{(1)},j})+\frac{w}{2}\|\hat{\boldsymbol{\theta}}_{\lambda^{(1)},j}\|_{2}^{2}\right)\\
 & \le & \frac{1}{2}\left\Vert y-\sum_{j=1}^{J}N_{T,j}\boldsymbol{\theta}_{j}^{*}\right\Vert _{T}^{2}+\sum_{j=1}^{J}\lambda_{j}^{(1)}\left(P_{j}(\boldsymbol{\theta}_{j}^{*})+\frac{w}{2}\|\boldsymbol{\theta}_{j}^{*}\|_{2}^{2}\right)+\sum_{j=1}^{J}\left(\lambda_{j}-\lambda_{j}^{(1)}\right)\max_{j=1:J}\left(P_{j}(\hat{\boldsymbol{\theta}}_{\lambda^{(1)},j})+\frac{w}{2}\|\hat{\boldsymbol{\theta}}_{\lambda^{(1)},j}\|_{2}^{2}\right)\\
 & \le & C+J\lambda_{max}\max_{j=1:J}\left(P_{j}(\hat{\boldsymbol{\theta}}_{\lambda^{(1)},j})+\frac{w}{2}\|\hat{\boldsymbol{\theta}}_{\lambda^{(1)},j}\|_{2}^{2}\right)
\end{eqnarray*}


where 
\[
C=\frac{1}{2}\left\Vert y-\sum_{j=1}^{J}N_{T,j}\boldsymbol{\theta}_{j}^{*}\right\Vert _{T}^{2}+\lambda_{max}\sum_{j=1}^{J}\left(P_{j}(\boldsymbol{\theta}_{j}^{*})+\frac{w}{2}\|\boldsymbol{\theta}_{j}^{*}\|_{2}^{2}\right)
\]


Note that by the basic inequality, we also know that 
\begin{eqnarray*}
\lambda_{min}\left(\max_{j=1:J}P_{j}(\hat{\boldsymbol{\theta}}_{\lambda^{(1)},j})+\frac{w}{2}\|\hat{\boldsymbol{\theta}}_{\lambda^{(1)},j}\|_{2}^{2}\right) & \le & \frac{1}{2}\left\Vert y-\sum_{j=1}^{J}N_{T,j}\boldsymbol{\theta}_{j}^{*}\right\Vert _{T}^{2}+\sum_{j=1}^{J}\lambda_{j}^{(1)}\left(P_{j}(\boldsymbol{\theta}_{j}^{*})+\frac{w}{2}\|\boldsymbol{\theta}_{j}^{*}\|_{2}^{2}\right)\\
 & \le & C
\end{eqnarray*}


Plugging all this back into the previous inequalities, we get that
\[
\frac{1}{2}\left\Vert y-\sum_{j=1}^{J}N_{T,j}(\hat{\boldsymbol{\theta}}_{\lambda^{(1)},j}+\hat{m}_{j}(\boldsymbol{\lambda})\boldsymbol{\beta}_{j})\right\Vert _{T}^{2}\le C\left(1+\frac{J\lambda_{max}}{\lambda_{min}}\right)\implies\left\Vert y-\sum_{j=1}^{J}N_{T,j}(\hat{\boldsymbol{\theta}}_{\lambda^{(1)},j}+\hat{m}_{j}(\boldsymbol{\lambda})\boldsymbol{\beta}_{j})\right\Vert _{T}\le\sqrt{2C\left(1+\frac{\lambda_{max}J}{\lambda_{min}}\right)}
\]


and for all $j=1,...,J$
\[
\frac{\lambda_{j}w}{2}\left\Vert \boldsymbol{\hat{\theta}}_{\lambda^{(1)},j}+\hat{m}_{j}(\boldsymbol{\lambda})\boldsymbol{\beta}_{j}\right\Vert ^{2}\le\left(1+\frac{J\lambda_{max}}{\lambda_{min}}\right)C\implies\left\Vert \boldsymbol{\hat{\theta}}_{\lambda^{(1)},j}+\hat{m}_{j}(\boldsymbol{\lambda})\boldsymbol{\beta}_{j}\right\Vert \le\sqrt{\left(1+\frac{J\lambda_{max}}{\lambda_{min}}\right)\frac{2C}{\lambda_{min}w}}
\]


Hence
\begin{eqnarray*}
\left\Vert \left.\frac{\partial}{\partial m_{j}}P_{j}(\hat{\boldsymbol{\theta}}_{\lambda^{(1)},j}+m_{j}\boldsymbol{\beta}_{j})\right|_{m_{j}=\hat{m}_{j}(\lambda)}\right\Vert  & \le & \frac{\|\boldsymbol{\beta}_{k}\|}{\lambda_{min}}\left(B\sqrt{2C\left(1+\frac{J\lambda_{max}}{\lambda_{min}}\right)}+\lambda_{max}w\sqrt{\left(1+\frac{J\lambda_{max}}{\lambda_{min}}\right)\frac{2C}{\lambda_{min}w}}\right)\\
 & = & \frac{1}{\lambda_{min}}\left(B+\lambda_{max}\sqrt{\frac{w}{\lambda_{min}}}\right)\sqrt{\left(1+\frac{J\lambda_{max}}{\lambda_{min}}\right)2C}\|\boldsymbol{\beta}_{k}\|
\end{eqnarray*}


Then apply the Lemma for additive parametric models. From above, we
can plug in 
\[
K=\frac{1}{\lambda_{min}}\left(B+\lambda_{max}\sqrt{\frac{w}{\lambda_{min}}}\right)\sqrt{\left(1+\frac{J\lambda_{max}}{\lambda_{min}}\right)2C}
\]


The ``moreover'' statement follows from the fact that for any point
$\boldsymbol{x}$, we have 
\begin{eqnarray*}
\left|\sum_{j=1}^{J}\hat{g}_{j}(x_{j}|\boldsymbol{\lambda}^{(1)})-\hat{g}_{j}(x_{j}|\boldsymbol{\lambda}^{(2)})\right| & = & \left|\sum_{j=1}^{J}\sum_{i=1}^{B}\left(\hat{\theta}_{\lambda^{(1)},j,i}-\hat{\theta}_{\lambda^{(2)},j,i}\right)N_{j,i}(x_{j})\right|\\
 & \le & \sum_{j=1}^{J}\sum_{i=1}^{B}\left|\left(\hat{\theta}_{\lambda^{(1)},j,i}-\hat{\theta}_{\lambda^{(2)},j,i}\right)N_{j,i}(x_{j})\right|\\
 & \le & \sum_{j=1}^{J}\sum_{i=1}^{B}\left|\hat{\theta}_{\lambda^{(1)},j,i}-\hat{\theta}_{\lambda^{(2)},j,i}\right|\\
 & \le & \sum_{j=1}^{J}\|\hat{\boldsymbol{\theta}}_{\lambda^{(1)},j}-\hat{\boldsymbol{\theta}}_{\lambda^{(2)},j}\|_{1}\\
 & \le & \sqrt{B}\sum_{j=1}^{J}\|\hat{\boldsymbol{\theta}}_{\lambda^{(1)},j}-\hat{\boldsymbol{\theta}}_{\lambda^{(2)},j}\|_{2}
\end{eqnarray*}


where the second inequality uses the fact that normalized B-splines
have value at most 1. Therefore

\begin{eqnarray*}
\left\Vert \sum_{j=1}^{J}\hat{g}_{j}(x_{j}|\lambda^{(1)})-\hat{g}_{j}(x_{j}|\lambda^{(2)})\right\Vert _{\infty} & \le & \sqrt{B}\sum_{j=1}^{J}\left\Vert \hat{\boldsymbol{\theta}}_{\lambda^{(1)},j}-\hat{\boldsymbol{\theta}}_{\lambda^{(2)},j}\right\Vert 
\end{eqnarray*}

\end{document}
