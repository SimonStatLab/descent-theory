%% LyX 2.1.3 created this file.  For more info, see http://www.lyx.org/.
%% Do not edit unless you really know what you are doing.
\documentclass[english]{article}
\usepackage[T1]{fontenc}
\usepackage[latin9]{inputenc}
\usepackage{geometry}
\geometry{verbose,tmargin=3cm,bmargin=3cm,lmargin=3cm,rmargin=3cm}
\usepackage{esint}
\usepackage{babel}
\begin{document}

\subsection*{Sobolev penalty:univariate}

Given a function $h$, the Sobolev penalty for $h$ is

\[
P(h)=\int(h^{(r)}(x))^{2}dx
\]


Suppose $\sup_{g}\|g\|_{\infty}\le G$.

We shall suppose for simplicity that the domain is $[0,1]$.

Suppose we have the function class (so no additional ridge penalty)
\[
\hat{\mathcal{G}}(T)=\left\{ \hat{g}(\cdot|\lambda)=\arg\min_{g\in\mathcal{G}}\frac{1}{2}\|y-g\|_{T}^{2}+\lambda P(g):\lambda\in\Lambda\right\} 
\]


Using the logic in Example 9.3.2 in Vandegeer, we can express any
function in $\mathcal{G}$ as 
\[
f+g
\]
where 
\[
g=\sum_{k=1}^{r}\alpha_{k}\psi_{k},f=\int_{0}^{1}\beta_{u}\tilde{\phi}_{u}
\]


where $\langle\psi_{k},\tilde{\phi}_{u}\rangle_{T}=0$ and $P(\psi_{k})=0$.

Suppose the observations were drawn from $y=f^{*}(x)+g^{*}(x)+\epsilon$
where $\epsilon$ are independent sub-gaussian random variables.

Now we have the function class

\[
\hat{\mathcal{G}}(T)=\left\{ \hat{g}(\cdot|\lambda),\hat{f}(\cdot|\lambda)=\arg\min_{g\in\mathcal{G}}\frac{1}{2}\|y-(f+g)\|_{T}^{2}+\lambda P(f):\lambda\in\Lambda,g=\sum_{k=1}^{r}\alpha_{k}\psi_{k},f=\int_{0}^{1}\beta_{u}\tilde{\phi}_{u}\right\} 
\]


We will show that
\begin{eqnarray*}
\left\Vert \left(\hat{g}(\cdot|\lambda^{(1)})+\hat{f}(\cdot|\lambda^{(1)})\right)-\left(\hat{g}(\cdot|\lambda^{(2)})+\hat{f}(\cdot|\lambda^{(2)})\right)\right\Vert _{\infty} & \le & |\lambda^{(1)}-\lambda^{(2)}|n^{\tau_{min}}\sqrt{\frac{n^{\tau_{min}}}{2}\|\epsilon\|_{T}^{2}+P(f^{*})}G
\end{eqnarray*}



\subsubsection*{Proof}

First by Vandegeer Example 9.3.2, we know that 
\[
\hat{g}(\cdot|\lambda)=\arg\min_{g=\sum\alpha_{k}\psi_{k}}-2\langle\epsilon,g-g^{*}\rangle_{T}+\|g-g^{*}\|_{T}^{2}
\]


\[
\hat{f}(\cdot|\lambda)=\arg\min_{f=\int_{0}^{1}\beta_{u}\tilde{\phi}_{u}}-2\langle\epsilon,f-f^{*}\rangle_{T}+\|f-f^{*}\|_{T}^{2}+\lambda P(f)
\]


So $\hat{g}(\cdot|\lambda)$ is actually independent of $\lambda$
and is therefore constant. We will just denote it $\hat{g}$ from
now on.

Now consider 
\[
h=c\left(\hat{f}(\cdot|\lambda^{(1)})-\hat{f}(\cdot|\lambda^{(2)})\right)
\]
 where $c$ is some constant s.t. $P(h)=1$.

We can assume that $P(h)\ne0$. Otherwise, if 
\begin{eqnarray*}
P\left(\hat{f}(\cdot|\lambda^{(1)})-\hat{f}(\cdot|\lambda^{(2)})\right) & = & 0
\end{eqnarray*}
then we know that
\[
\hat{f}(\cdot|\lambda^{(1)})-\hat{f}(\cdot|\lambda^{(2)})\in span\left\{ \psi_{k}\right\} _{k=1}^{r}
\]


This is true if and only if $\hat{f}(\cdot|\lambda^{(1)})\equiv\hat{f}(\cdot|\lambda^{(2)})$
(by the fact that the function spaces are orthogonal).

Consider the optimization problem

\[
\hat{m}_{h}(\lambda)=\arg\min_{m}\frac{1}{2}\|y-(\hat{g}+\hat{f}(\cdot|\lambda^{(1)})+mh)\|_{T}^{2}+\lambda P\left(\hat{f}(\cdot|\lambda^{(1)})+mh\right)
\]


By implicit differentiation of the KKT conditions, we get
\begin{eqnarray*}
\left.\frac{\partial}{\partial\lambda}\hat{m}_{h}(\lambda)\right|_{\lambda=\lambda} & = & \left.-\left(\|h\|_{T}^{2}+\lambda\frac{\partial^{2}}{\partial m^{2}}P\left(\hat{f}(\cdot|\lambda^{(1)})+mh\right)\right)^{-1}\frac{\partial}{\partial m}P\left(\hat{f}(\cdot|\lambda^{(1)})+mh\right)\right|_{m=\hat{m}_{h}(\lambda)}
\end{eqnarray*}


Then the first multiplicand is bounded by 
\begin{eqnarray*}
\left|\|h\|_{T}^{2}+\lambda\frac{\partial^{2}}{\partial m^{2}}P\left(\hat{f}(\cdot|\lambda^{(1)})+mh\right)\right|^{-1} & \le & n^{\tau_{min}}\frac{\partial^{2}}{\partial m^{2}}P\left(\hat{f}(\cdot|\lambda^{(1)})+mh\right)^{-1}\\
 & = & \frac{n^{\tau_{min}}}{2P(h)}
\end{eqnarray*}


The equality follows from the Lemma Sobolev Facts (see below).

From the Lemma Sobolev Facts and by the fact that $P(h)=1$, we have
\begin{eqnarray*}
\left|\frac{\partial}{\partial\lambda}\hat{m}_{h}(\lambda)\right|_{\lambda=\lambda} & \le & \frac{n^{\tau_{min}}}{P(h)}\sqrt{P\left(\hat{f}(\cdot|\lambda^{(1)})+\hat{m}_{h}(\lambda)h\right)P(h)}\\
 & = & n^{\tau_{min}}\sqrt{P\left(\hat{f}(\cdot|\lambda^{(1)})+\hat{m}_{h}(\lambda)h\right)}
\end{eqnarray*}


By the definition of $\hat{m}_{h}(\lambda)$ and $\hat{f}(\cdot|\lambda^{(1)})$,
we have that
\begin{eqnarray*}
\lambda P\left(\hat{f}(\cdot|\lambda^{(1)})+\hat{m}_{h}(\lambda)h\right) & \le & \frac{1}{2}\|y-(\hat{g}+\hat{f}(\cdot|\lambda^{(1)}))\|_{T}^{2}+\lambda P\left(\hat{f}(\cdot|\lambda^{(1)})\right)\\
 & = & \frac{1}{2}\|y-(\hat{g}+\hat{f}(\cdot|\lambda^{(1)}))\|_{T}^{2}+\lambda^{(1)}P\left(\hat{f}(\cdot|\lambda^{(1)})\right)+\left(\lambda-\lambda^{(1)}\right)P\left(\hat{f}(\cdot|\lambda^{(1)})\right)\\
 & \le & \frac{1}{2}\|y-(g^{*}+f^{*})\|_{T}^{2}+\lambda^{(1)}P\left(f^{*}\right)+\left(\lambda-\lambda^{(1)}\right)P\left(\hat{f}(\cdot|\lambda^{(1)})\right)
\end{eqnarray*}


In addition, by definition of $\hat{f}(\cdot|\lambda^{(1)})$, we
have 
\[
P\left(\hat{f}(\cdot|\lambda^{(1)})\right)\le\frac{1}{2\lambda^{(1)}}\|y-(g^{*}+f^{*})\|_{T}^{2}+P\left(f^{*}\right)
\]


Combining the two inequalities above, we have
\begin{eqnarray*}
\lambda P\left(\hat{f}(\cdot|\lambda^{(1)})+\hat{m}_{h}(\lambda)h\right) & \le & \frac{1}{2}\|\epsilon\|_{T}^{2}+\lambda^{(1)}P\left(f^{*}\right)+\left(\lambda-\lambda^{(1)}\right)\left(\frac{1}{2\lambda^{(1)}}\|\epsilon\|_{T}^{2}+P\left(f^{*}\right)\right)\\
 & \le & \frac{\lambda}{2\lambda^{(1)}}\|\epsilon\|_{T}^{2}+\lambda P\left(f^{*}\right)
\end{eqnarray*}


Therefore 
\[
P\left(\hat{f}(\cdot|\lambda^{(1)})+\hat{m}_{h}(\lambda)h\right)\le\frac{n^{\tau_{min}}}{2}\|\epsilon\|_{T}^{2}+P(f^{*})
\]


Then by the MVT, we have 
\begin{eqnarray*}
\|\hat{f}(\cdot|\lambda^{(1)})-\hat{f}(\cdot|\lambda^{(2)})\|_{\infty} & = & \|\left(m_{h}(\lambda^{(2)})-m_{h}(\lambda^{(1)})\right)h\|_{\infty}\\
 & = & \|h\|_{\infty}\left|\lambda^{(2)}-\lambda^{(1)}\right|\left.\frac{\partial}{\partial\lambda}\hat{m}_{h}(\lambda)\right|_{\lambda\in[\lambda^{(1)},\lambda^{(2)}]}\\
 & \le & G\left|\lambda^{(1)}-\lambda^{(2)}\right|\left(\sup_{\lambda\in[\lambda^{(1)},\lambda^{(2)}]}\left|\frac{\partial}{\partial\lambda}\hat{m}_{h}(\lambda)\right|_{\lambda=\lambda}\right)\\
 & \le & \left|\lambda^{(1)}-\lambda^{(2)}\right|Gn^{\tau_{min}}\sqrt{\frac{n^{\tau_{min}}}{2}\|\epsilon\|_{T}^{2}+P(f^{*})}
\end{eqnarray*}



\subsection*{Sobolev penalty: multivariate}

The function class of interest

\[
\hat{\mathcal{G}}(T)=\left\{ \left\{ \hat{g}_{j}(\cdot|\lambda),\hat{f}_{j}(\cdot|\lambda)\right\} =\arg\min_{g\in\mathcal{G}}\frac{1}{2}\|y-\sum_{j=1}^{J}g_{j}(x_{j})\|_{T}^{2}+\sum_{j=1}^{J}\lambda_{j}P(g_{j}):\lambda\in\Lambda\right\} 
\]


We can show that

\begin{eqnarray*}
\|\hat{f}_{\ell}(\cdot|\lambda^{(1)})-\hat{f}_{\ell}(\cdot|\lambda^{(2)})\|_{\infty} & \le & G\left\Vert \lambda^{(1)}-\lambda^{(2)}\right\Vert \frac{n^{\tau_{min}}}{2}\sqrt{\frac{n^{\tau_{min}}+Jn^{\tau_{max}+2\tau_{min}}}{2}\|\epsilon\|_{T}^{2}+n^{\tau_{max}+2\tau_{min}}\sum_{j=1}^{J}P\left(f_{j}^{*}\right)}
\end{eqnarray*}


A second approach gives a different bound:

\begin{eqnarray*}
\|\hat{f}_{\ell}(\cdot|\lambda^{(1)})-\hat{f}_{\ell}(\cdot|\lambda^{(2)})\|_{\infty} & \le & \left\Vert \lambda^{(1)}-\lambda^{(2)}\right\Vert \frac{G^{2}(2G+\|\epsilon\|_{T})n^{3\tau_{min}}}{4}
\end{eqnarray*}



\subsubsection*{Proof}

First by Vandegeer Example 9.3.2, we know that 
\[
\left\{ \hat{g}_{j}(\cdot|\lambda)\right\} _{j=1}^{J}=\arg\min_{g_{j}=\sum\alpha_{k}\psi_{k}}-2\langle\epsilon,\sum_{j=1}^{J}g_{j}-g_{j}^{*}\rangle_{T}+\|\sum_{j=1}^{J}g_{j}-g_{j}^{*}\|_{T}^{2}
\]


\[
\left\{ \hat{f}_{j}(\cdot|\lambda)\right\} _{j=1}^{J}=\arg\min_{f_{j}=\int_{0}^{1}\beta_{u}\tilde{\phi}_{u}}-2\langle\epsilon,\sum_{j=1}^{J}f_{j}-f_{j}^{*}\rangle_{T}+\|\sum_{j=1}^{J}f_{j}-f_{j}^{*}\|_{T}^{2}+\sum_{j=1}^{J}\lambda_{j}P(f_{j})
\]


For every $j=1:J$, we again notice that $\hat{g}_{j}(\cdot|\lambda)$
is independent of $\lambda$. We will just denote it $\hat{g}_{j}$
from now on.

For every $j=1:J$, define functions 
\[
h_{j}=c\left(\hat{f}_{j}(\cdot|\lambda^{(1)})-\hat{f}_{j}(\cdot|\lambda^{(2)})\right)
\]
 where $c$ is some constant s.t. $P(h_{j})=1$.

We can assume that $P(h_{j})\ne0$. Otherwise, if 
\begin{eqnarray*}
P\left(\hat{f}_{j}(\cdot|\lambda^{(1)})-\hat{f}_{j}(\cdot|\lambda^{(2)})\right) & = & 0
\end{eqnarray*}
then we know that
\[
\hat{f}_{j}(\cdot|\lambda^{(1)})-\hat{f}_{j}(\cdot|\lambda^{(2)})\in span\left\{ \psi_{k}\right\} _{k=1}^{r}
\]


This is true if and only if $\hat{f}_{j}(\cdot|\lambda^{(1)})\equiv\hat{f}_{j}(\cdot|\lambda^{(2)})$
(by the fact that the function spaces are orthogonal).

Now consider the optimization problem

\[
\left\{ \hat{m}_{j}(\lambda,h)\right\} _{j=1}^{J}=\arg\min_{m_{j}}\frac{1}{2}\|y-\sum_{j=1}^{J}(\hat{g}_{j}+\hat{f}_{j}(\cdot|\lambda^{(1)})+m_{j}h_{j})\|_{T}^{2}+\sum_{j=1}^{J}\lambda_{j}P\left(\hat{f}_{j}(\cdot|\lambda^{(1)})+m_{j}h_{j}\right)
\]


(If $h_{j}\equiv0$, then set $m_{j}=0$ as a constant.) For simplicity,
we will assume $h_{j}\ne0$.

The KKT conditions give us for all $\ell=1:J$
\[
\left.\left\langle h_{\ell},y-\left(\sum_{j=1}^{J}\hat{g}_{j}(\cdot|\lambda^{(1)})+\hat{f}_{j}(\cdot|\lambda^{(1)})+\hat{m}_{j}(\lambda,h)h_{j}\right)\right\rangle _{T}+\lambda_{\ell}\frac{\partial}{\partial m_{\ell}}P\left(\hat{f}_{\ell}(\cdot|\lambda^{(1)})+m_{\ell}h\right)\right|_{m_{\ell}=\hat{m}_{\ell}(\lambda,h)}=0
\]


For all $k=1:J$, by implicit differentiation of the KKT conditions
with respect to $\lambda_{k}$, we get 
\begin{eqnarray*}
\left\langle h_{\ell},y-\sum_{j=1}^{J}h_{j}\frac{\partial}{\partial\lambda_{k}}\hat{m}_{j}(\lambda,h)\right\rangle _{T}+\lambda_{\ell}\frac{\partial^{2}}{\partial m_{\ell}^{2}}P\left(\hat{f}_{\ell}(\cdot|\lambda^{(1)})+m_{\ell}h\right)\frac{\partial}{\partial\lambda_{k}}\hat{m}_{\ell}(\lambda,h)\\
+1[\ell=k]\frac{\partial}{\partial m_{\ell}}P\left(\hat{f}_{\ell}(\cdot|\lambda^{(1)})+m_{\ell}h\right) & = & 0
\end{eqnarray*}


Define the following matrices 
\[
S:S_{ij}=\langle h_{j},h_{\ell}\rangle_{T}
\]
\[
D_{1}=diag\left(\lambda_{\ell}\frac{\partial^{2}}{\partial m_{\ell}^{2}}P\left(\hat{f}_{\ell}(\cdot|\lambda^{(1)})+\hat{m}_{\ell}(\lambda)h_{\ell}\right)\right)
\]


\[
D_{3}=diag\left(\frac{\partial}{\partial m_{\ell}}P\left(\hat{f}_{\ell}(\cdot|\lambda^{(1)})+\hat{m}_{\ell}(\lambda)h_{\ell}\right)\right)
\]


\[
M=\left(\begin{array}{cccc}
\frac{\partial\hat{m}_{1}(\lambda)}{\partial\lambda} & \frac{\partial\hat{m}_{2}(\lambda)}{\partial\lambda} & ... & \frac{\partial\hat{m}_{J}(\lambda)}{\partial\lambda}\end{array}\right)
\]


From the implicit differentiation equations, we have the following
system of equations: 
\[
M=D_{3}\left(S+D_{1}\right)^{-1}
\]


We know that $S$ is a PSD matrix (since it can be written as $S=HH^{T}$
where $H_{j}=h_{j}$ evaluated at covariates $T$).

We are interested in bounding $\nabla_{\lambda}\hat{m}_{\ell}(\lambda,h)$,
which is the $\ell$-th column of $M$ has norm. By Lemma PSD\_Matrix\_Inverse
(see additive\_models.pdf), we know that 
\begin{eqnarray*}
\|\nabla_{\lambda}\hat{m}_{\ell}(\lambda,h)\| & = & \|Me_{\ell}\|\\
 & = & \|D_{3}\left(S+D_{1}\right)^{-1}e_{\ell}\|\\
 & \le & \|D_{3}D_{1}^{-1}e_{\ell}\|\\
 & = & \left|\frac{\partial}{\partial m_{\ell}}P\left(\hat{f}_{\ell}(\cdot|\lambda^{(1)})+\hat{m}_{\ell}(\lambda)h_{\ell}\right)\right|\left|\lambda_{\ell}\frac{\partial^{2}}{\partial m_{\ell}^{2}}P\left(\hat{f}_{\ell}(\cdot|\lambda^{(1)})+\hat{m}_{\ell}(\lambda)h_{\ell}\right)\right|^{-1}
\end{eqnarray*}


By Lemma Sobolev Facts (below), we have 
\[
\frac{\partial^{2}}{\partial m_{\ell}^{2}}P\left(\hat{f}_{\ell}(\cdot|\lambda^{(1)})+\hat{m}_{\ell}(\lambda)h_{\ell}\right)=2P(h_{\ell})=2
\]


Also by Lemma Sobolev Facts (below), we note that
\begin{eqnarray*}
\left|\frac{\partial}{\partial m_{\ell}}P\left(\hat{f}_{\ell}(\cdot|\lambda^{(1)})+\hat{m}_{\ell}(\lambda)h_{\ell}\right)\right| & \le & 2\sqrt{P\left(\hat{f}_{\ell}(\cdot|\lambda^{(1)})+\hat{m}_{\ell}(\lambda)h_{\ell}\right)P(h_{\ell})}\\
 & = & 2\sqrt{P\left(\hat{f}_{\ell}(\cdot|\lambda^{(1)})+\hat{m}_{\ell}(\lambda)h_{\ell}\right)}
\end{eqnarray*}


By the definition of $\hat{m}_{\ell}(\lambda)$ and $\hat{f}(\cdot|\lambda^{(1)})$,
we have
\begin{eqnarray*}
\lambda_{\ell}P\left(\hat{f}_{\ell}(\cdot|\lambda^{(1)})+\hat{m}_{\ell}(\lambda)h_{\ell}\right) & \le & \frac{1}{2}\|y-(\hat{g}+\hat{f}(\cdot|\lambda^{(1)}))\|_{T}^{2}+\sum_{j=1}^{J}\lambda_{j}P\left(\hat{f}_{j}(\cdot|\lambda^{(1)})\right)\\
 & = & \frac{1}{2}\|y-(g^{*}+f^{*})\|_{T}^{2}+\sum_{j=1}^{J}\lambda_{j}^{(1)}P\left(f_{j}^{*}\right)+\sum_{j=1}^{J}\left(\lambda_{j}-\lambda_{j}^{(1)}\right)P\left(\hat{f}_{j}(\cdot|\lambda^{(1)})\right)\\
 & \le & \frac{1}{2}\|\epsilon\|_{T}^{2}+\lambda_{max}\sum_{j=1}^{J}P\left(f_{j}^{*}\right)+J\lambda_{max}\left[\max_{j=1:J}P\left(\hat{f}_{j}(\cdot|\lambda^{(1)})\right)\right]
\end{eqnarray*}


By definition of $\hat{f}_{j}(\cdot|\lambda^{(1)})$ , we know 
\[
\max_{j=1:J}P\left(\hat{f}_{j}(\cdot|\lambda^{(1)})\right)\le\frac{1}{\lambda_{min}}\left(\frac{1}{2}\|\epsilon\|_{T}^{2}+\sum_{j=1}^{J}\lambda_{j}^{(1)}P\left(f_{j}^{*}\right)\right)
\]


So 
\[
P\left(\hat{f}_{\ell}(\cdot|\lambda^{(1)})+m_{\ell}h_{\ell}\right)\le\frac{n^{\tau_{min}}+Jn^{\tau_{max}+2\tau_{min}}}{2}\|\epsilon\|_{T}^{2}+n^{\tau_{max}+2\tau_{min}}\sum_{j=1}^{J}P\left(f_{j}^{*}\right)
\]


Then by the MVT, we have for some $\alpha\in(0,1)$ 
\begin{eqnarray*}
\|\hat{f}_{\ell}(\cdot|\lambda^{(1)})-\hat{f}_{\ell}(\cdot|\lambda^{(2)})\|_{\infty} & = & \|\left(\hat{m}_{\ell}(\lambda^{(2)},h)-\hat{m}_{\ell}(\lambda^{(1)},h)\right)h_{\ell}\|_{\infty}\\
 & = & \|h_{\ell}\|_{\infty}\langle\lambda^{(1)}-\lambda^{(2)},\nabla_{\lambda}\hat{m}_{\ell}(\lambda,h)\rangle_{\lambda=\alpha\lambda^{(1)}+(1-\alpha)\lambda^{(2)}}\\
 & \le & G\left\Vert \lambda^{(1)}-\lambda^{(2)}\right\Vert \left\Vert \nabla_{\lambda}\hat{m}_{\ell}(\lambda,h)\right\Vert \\
 & \le & G\left\Vert \lambda^{(1)}-\lambda^{(2)}\right\Vert \frac{n^{\tau_{min}}}{2}\sqrt{\frac{n^{\tau_{min}}+Jn^{\tau_{max}+2\tau_{min}}}{2}\|\epsilon\|_{T}^{2}+n^{\tau_{max}+2\tau_{min}}\sum_{j=1}^{J}P\left(f_{j}^{*}\right)}
\end{eqnarray*}


\textbf{A second approach:}

By the KKT conditions, we also know that 
\begin{eqnarray*}
\left|\frac{\partial}{\partial m_{\ell}}P\left(\hat{f}_{\ell}(\cdot|\lambda^{(1)})+\hat{m}_{\ell}(\lambda)h_{\ell}\right)\right| & = & \frac{1}{\lambda_{\ell}}\left|\left\langle h_{\ell},y-\left(\sum_{j=1}^{J}\hat{g}_{j}(\cdot|\lambda^{(1)})+\hat{f}_{j}(\cdot|\lambda^{(1)})+\hat{m}_{j}(\lambda,h)h_{j}\right)\right\rangle _{T}\right|\\
 & \le & \frac{1}{\lambda_{min}}\|h_{\ell}\|_{T}\left\Vert y-\left(\sum_{j=1}^{J}\hat{g}_{j}(\cdot|\lambda^{(1)})+\hat{f}_{j}(\cdot|\lambda^{(1)})+\hat{m}_{j}(\lambda,h)h_{j}\right)\right\Vert _{T}\\
 & \le & G(2G+\|\epsilon\|_{T})n^{\tau_{min}}
\end{eqnarray*}


Hence 
\[
\|\nabla_{\lambda}\hat{m}_{\ell}(\lambda,h)\|\le G(2G+\|\epsilon\|_{T})n^{2\tau_{min}}\frac{1}{2}
\]


Then by the MVT, we have for some $\alpha\in(0,1)$ 
\begin{eqnarray*}
\|\hat{f}_{\ell}(\cdot|\lambda^{(1)})-\hat{f}_{\ell}(\cdot|\lambda^{(2)})\|_{\infty} & = & \|\left(\hat{m}_{\ell}(\lambda^{(2)},h)-\hat{m}_{\ell}(\lambda^{(1)},h)\right)h_{\ell}\|_{\infty}\\
 & = & \|h_{\ell}\|_{\infty}\langle\lambda^{(1)}-\lambda^{(2)},\nabla_{\lambda}\hat{m}_{\ell}(\lambda,h)\rangle_{\lambda=\alpha\lambda^{(1)}+(1-\alpha)\lambda^{(2)}}\\
 & \le & G\left\Vert \lambda^{(1)}-\lambda^{(2)}\right\Vert \left\Vert \nabla_{\lambda}\hat{m}_{\ell}(\lambda,h)\right\Vert \\
 & \le & \left\Vert \lambda^{(1)}-\lambda^{(2)}\right\Vert G^{2}(2G+\|\epsilon\|_{T})n^{3\tau_{min}}\frac{1}{4}
\end{eqnarray*}



\subsubsection*{Lemma: Sobolev Facts}

For any function $h$, we have
\begin{eqnarray*}
\left|\frac{\partial}{\partial m}P(g+mh)\right| & = & \left|2\int(g^{(r)}(x)+mh^{(r)}(x))h^{(r)}(x)dx\right|\\
 & \le & 2\sqrt{P(g+mh)P(h)}
\end{eqnarray*}


and 
\[
\frac{\partial^{2}}{\partial m^{2}}P(g+mh)=2\int(h^{(r)}(x))^{2}dx=2P(h)
\]

\end{document}
